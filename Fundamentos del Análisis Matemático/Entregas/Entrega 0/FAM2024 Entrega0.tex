\documentclass[a4paper,oneside,11pt,leqno]{article}

\usepackage[latin1]{inputenc}
\usepackage[spanish]{babel}

\usepackage{amsfonts}
\usepackage{amsmath}
\usepackage{fancyhdr}
\usepackage{epic}
\usepackage{eepic}
\usepackage{amssymb}
\usepackage{hyperref}
\usepackage{fancybox}

\usepackage{color}
\usepackage{graphicx}
\usepackage{epsfig}
\usepackage{epstopdf}
\usepackage{esint}


\newcommand{\red}{\textcolor{red}}
\newcommand{\blue}{\textcolor{blue}}


\textwidth = 16truecm 
\textheight = 24truecm
\oddsidemargin =-20pt
\evensidemargin = 5pt
\topmargin=-1truecm

\begin{document}
%\thispagestyle{empty}


\begin{center}{\bf FUNDAMENTOS DE AN\'ALISIS MATEM\'ATICO-FAM}  \\
Departamento de Matem\'aticas\\
Universidad Aut\'onoma de Madrid\\
\end{center}
\hfill \red{\it A entregar el jueves, 19 de septiembre}


\vskip 6pt \hrule

\vskip 3mm
\noindent{{\bf Entrega 0}: Redacta un breve ensayo en el que describas tus conocimientos previos en An\'alisis Matem\'atico y tus expectativas tras elegir esta asignatura del M\'aster. Adem\'as, a modo de ejercicio sencillo, me gustar\'ia que enunciaras un resultado de los que hayas visto en otros cursos de An\'alisis, que te haya impactado, ya sea por su dificultad, por su belleza o por cualquier otro motivo.} \footnote{Tama\~no esperado de este ensayo: entre 1 y $1\frac 12$ pg.}
\vskip 3mm \hrule

\vskip 5mm

\noindent{\bf  Nombre}:

\vskip 5mm

\noindent{\bf  Esquema sugerido}:


\vskip 5mm

\begin{enumerate}

\item[1.-] {  Mi experiencia previa en An\'alisis Matem\'atico}: 

\item[2.-] { Mis expectativas}: 

\item[2.-] { Mi resultado favorito en An\'alisis}: 

\item[3.-] { Bibliograf\'ia (opcional)}: \\


\end{enumerate}

\end{document}

