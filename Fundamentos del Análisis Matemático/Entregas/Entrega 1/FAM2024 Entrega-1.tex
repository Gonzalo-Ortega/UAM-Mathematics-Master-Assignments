%%%%%%%%%%%%%%%%%%

\documentclass[11pt,a4paper,draft]{article}

%%%%%%%%%%%%%%%%%%

\usepackage{latexsym}
\usepackage{amssymb}
\usepackage{amsmath}
\usepackage{amsfonts}
\usepackage{graphicx}
\usepackage{epsfig}
\usepackage{epstopdf}
\usepackage{esint}

\usepackage{mathtools} %for \stackrel

%%%%%%%%%%%%%%%%%%

\usepackage{color}
\newcommand{\red}{\textcolor{red}}
\newcommand{\blue}{\textcolor{blue}}

%%%%%%%%%%%%%%%%%%

\usepackage[spanish]{babel}
\usepackage[latin1]{inputenc}

%%%%%%%%%%%%%%%%%%

\addtolength{\topmargin}{-3.5cm} \addtolength{\oddsidemargin}{-2cm}
\addtolength{\textheight}{+5cm} \addtolength{\textwidth}{+4cm}


\parindent=0mm
\parskip 0mm

\thispagestyle{empty}


%%%%%%%%%%%%%%%%%%%%%%%%%%%%%%%%%%%
%%%%%%%%%%%%%%%%%%%%%%%%%%%%%%%%%%%


\begin{document}
\hfill (\red{\it A entregar el 26.09.2024})
\hrule\hrule
\vspace{1mm}

\noindent {\bf Fundamentos de An\'alisis Matem\'atico, MMA 2023-24.
\hfill{Entrega 1}}

\vspace{1mm}

 \noindent {\bf Nombre y APELLIDOS}: 
\vspace{2mm}

\hrule\hrule

\vspace{2mm}

\

{\bf 1)} Se dice que la funci\'on $f:\mathbb R^N \to \mathbb R$ es H\"older de orden $\alpha>0$ si existe una constante $C$ de forma que
$$
|f(x)-f(y)|\le C|x-y|^\alpha, \; \forall x,y \in \mathbb R^N. 
$$
Probar que si $f$ es H\"older de orden $\alpha$, con $\alpha>N$, entonces $f$ es constante. 

\vskip 5mm

 {\bf 2.}  En $\mathbb R^N$, si el conjunto $A$ no es medible Lebesgue, y  $s<N$, probar que  $\mathcal{H}^s_*(A)=\infty$.
 
\vskip 5mm

 \noindent {\bf 3.} Recordamos que la medida exterior de Lebesgue se define como
 \vskip -3mm
 $$
 m^*(E)=\inf\left\{\sum_{j\ge 1} \mbox{vol}(B_j): \{B_j\}_j \mbox{ cubrimiento por bolas de } A\right\}. 
 $$
  \vskip -3mm
  Definimos por otro lado la clase
  $$
  \mathcal{B}=\{A\subset \mathbb R: \forall \epsilon>0, \; \exists \mathcal{O},  \mbox{ abierto, tal que }\; A\subset \mathcal{O} \mbox{ y } \; m^*(\mathcal{O}\setminus A)<\epsilon\}. 
  $$
Probar: 
\begin{itemize}
\item  $\mathcal{B}$ es una $\sigma$-\'algebra en $\mathbb R^N$.
\item  $\mathcal{B}$ coincide con la $\sigma$-\'algebra $\mathcal A$ obtenida por el teorema de Caratheodory. 
\end{itemize}

\vskip 1mm
\hrule
\vskip 3mm

\noindent {\bf SOL.:} 
 
 \end{document}

%%%%%%%%%%%%%%%%%%%%%%%%%%%%%%%%%%%
%%%%%%%%%%%%%%%%%%%%%%%%%%%%%%%%%%%\quad 

,