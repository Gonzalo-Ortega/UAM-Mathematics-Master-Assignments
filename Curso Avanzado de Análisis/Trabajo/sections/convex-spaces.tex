\section{Espacios convexos}

La principal herramienta para trabajar en espacios métricos, y en particular en los espacios normados \cite{babb}, es la desigualdad triangular 
\begin{equation}
    \| x + y \| \leq \| x \| + \| y \|.
\end{equation}

\begin{definition}[Conjunto convexo]
    Un conjunto $ C \subseteq X $ es {\bf convexo} si para todo par de puntos $ x, y \in C $, $ t \in [0, 1] $, se tiene $ tx + (1-t)y \in C $.
\end{definition}

\begin{proposition}
    Sea $ p \colon X \to [0,\infty) $ una función con la propiedad de que para todo $ x \in X $ y para todo $ \lambda \in \R $ se tiene que $ p(\lambda x) = |\lambda| p(x) $. Dicha función satisface la desigualdad triangular si y solo si la bola  $B_p \coloneq \{ x \in X \colon p(x) \leq 1 \} $ es compacto.
\end{proposition}
\begin{proof}
    Si $ p $ satisface la desigualdad triangular, dados $ x, y \in B_p $, $ p(x) \leq 1 $ y  $ p(y) \leq 1 $, y tenemos que para todo $ t \in [0, 1] $,
    \begin{equation}
        p(tx + (1-t) y) \leq p(tx) + p((1-t)y) = tp(x) + (1-t)p(y) \leq t + (1-t) = 1.
    \end{equation}
    Por tanto, $ tx + (1-t)y \in B_p $ y $ B_p $ es convexa.

    Supongamos ahora que $ B_p $ es convexa. Para todo $ x, y \in X $ con $ p(x), p(y) \neq 0 $, definimos $ t \in (0, 1) $ como
    \begin{equation}
        t \coloneq \frac{p(x)}{p(x) + p(y)}, \text{ donde } 1 - t = \frac{p(y)}{p(x) + p(y)}.
    \end{equation}
    Tenemos que $ \frac{x}{p(x) } $ y $ \frac{x}{p(x) } $ están contenidos en $ B_p $ ya que, tomando el primer caso como ejemplo, 
    \begin{equation}
        p(\frac{x}{p(x)}) = \frac{p(x)}{p(x)} = 1.
    \end{equation}
    Ahora bien, como $ B_p $ es convexa, 
    \begin{equation}
        1 \geq p\left(t \frac{x}{p(x)} + (1-t) \frac{y}{p(y)}\right) = p\left( \frac{x}{p(x)+p(y)} + \frac{y}{p(x)+p(y)} \right).
    \end{equation}
    Por tanto, despejando tenemos
    \begin{equation}
        p(x)+p(y) \geq p(x+y).
    \end{equation}
\end{proof}

Dado un espacio normado $ (X, \| \cdot \|) $, por conveniencia a veces denotado simplemente $ X $, definimos la bola cerrada y la esfera unidad en $ X $, $ B(X) $ y $ S(X) $ respectivamente como
\begin{equation}
\begin{split}
    B(X) &\coloneq \{x \in X \colon \| x \| \leq 1 \}, \\
    S(X) &\coloneq \{x \in X \colon \| x \| = 1 \}.
\end{split}
\end{equation}

\begin{definition}[Espacio estrictamente convexo]
    Un espacio normado $ (X, \| \cdot \|) $ es {\bf estrictamente convexo} si para todo par de puntos $ x $ e $ y $ en la esfera unidad $ S(X) $ tales que el punto medio del segmento que los une esta también en la esfera unidad, i.e. $\| \frac{x+y}{2}\| $, se tiene $ x = y $.
\end{definition}

\begin{definition}[Espacio uniformemente convexo]
    Un espacio normado $ (X, \| \cdot \|) $ es {\bf uniformemente convexo} si para todo $ \varepsilon \in (0, 2] $, existe un $ \delta \in (0, 1) $ tal que para todo par $ x, y \in B(X) $ con $ \| x - y \| < \varepsilon $ se tiene $ \| \frac{x+y}{2} \| < \delta $.
\end{definition}

\begin{corollary} \label{coro:1}
    Todo espacio uniformemente convexo es estrictamente convexo.
\end{corollary}

\begin{proposition} \label{prop:hilbert-convex}
    Sea $ (X, \langle \cdot, \cdot \rangle) $ un espacio dotado de un producto escalar con norma $ \|x\| \coloneq \sqrt{\langle x, x \rangle} $. Entonces el espacio $ (X, \|\cdot\|) $ es un espacio uniformemente, y por tanto también estrictamente, convexo. 
\end{proposition}

Para ver que la implicación del Corolario \ref{coro:1} no funciona en sentido contrario, el siguiente ejemplo presenta una norma estrictamente convexa, pero no uniformemente convexa, construyéndose a partir de una combinación de normas en espacios de sucesiones.

\begin{example}[Ejemplo 7, \cite{babb}]
    Consideremos la norma en el espacio de sucesiones convergentes $ \ell^1 $ dada por la suma de las normas $ \ell^1 $ y $ \ell^2 $, es decir, dada una sucesión $ x \in \ell^1 $,
    \begin{equation}
        \| x \| \coloneq \| x \|_{\ell^1} + \| x \|_{\ell^2}.
    \end{equation}
    Gracias a la Proposición \ref{prop:hilbert-convex} sabemos que para todo $ x \neq y \in S(\ell^2) $,
    \begin{equation}
        \|x+y\|_{\ell^2} < \|x\|_{\ell^2} + \|y\|_{\ell^2}.
    \end{equation}
    Por tanto, haciendo también uso de la desigualdad triangular estándar en $ \ell^1 $ tenemos
    \begin{equation}
        \|x+y\| = \|x+y\|_{\ell^1} + \|x+y\|_{\ell^2} < \left( \|x\|_{\ell^1} + \|y\|_{\ell^1} \right) + \left( \|x\|_{\ell^2} + \|y\|_{\ell^2} \right) = \|x\| + \|y\| = 2.
    \end{equation}
    Por tanto, $ \ell_1 $ es estrictamente convexo. Sin embargo, definamos ahora las sucesiones
    \begin{align}
        x_{N, k} &=
        \begin{cases}
            1 & \text{si } k \leq N, \\
            0 & \text{en caso contrario},
        \end{cases}
        &
        y_{N, k} &=
        \begin{cases}
            1 & \text{si } N < k \leq 2N, \\
            0 & \text{en caso contrario}.
        \end{cases}
    \end{align}
    Por un lado tenemos que $ \|x\| = \|y\| = N + \sqrt{N} $ y que $ \|x_N - y_N\| = 2N + \sqrt{2N} $, ya que
    \begin{equation}
        x_{N, k} - y_N =
        \begin{cases}
            1 & \text{si } k \leq N, \\
            -1 & \text{si } N < k \leq 2N, \\
            0 & \text{si } k \geq 2N.
        \end{cases}
    \end{equation}
    Pero tenemos también que
    \begin{equation}
        \left\| \frac{x_N+y_N}{2} \right\| = \frac{2N}{2} + \frac{\sqrt{2N}}{2}.
    \end{equation}
    Dividiendo entre $ N + \sqrt{N} $ podemos hacer $ \left\| \frac{x_N+y_N}{2 (N + \sqrt{N})} \right\| $ tan cercano como queramos a $ 1 $, mientras que siempre tendremos $ \left\| \frac{x_N+y_N}{(N + \sqrt{N})} \right\| \geq \sqrt{2}$. Por tanto, $ (\ell^1, \| \cdot \|) $ no es uniformemente convexo.
\end{example}