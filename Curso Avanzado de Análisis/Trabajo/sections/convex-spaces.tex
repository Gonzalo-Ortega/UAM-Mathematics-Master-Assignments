\section{Espacios convexos}

La principal herramienta para trabajar en espacios métricos, y en particular en los espacios normados \cite{babb}, es la desigualdad triangular 
\begin{equation}
    \| x + y \| \leq \| x \| + \| y \|.
\end{equation}

\begin{definition}[Conjunto convexo]
    Un conjunto $ C \subseteq X $ es {\bf convexo} si para todo par de puntos $ x, y \in C $, $ t \in [0, 1] $, se tiene $ tx + (1-t)y \in C $.
\end{definition}

Toda norma que definamos, ha de cumplir la desigualdad triangular, y una buena caracterización de esta es dada mediante sus relación con los conjuntos convexos. Esto es, un conjunto convexo es el que contiene todos los segments entre dos puntos del conjunto cualquiera. Por tanto, un conjunto es convexo si y solo si la bula unidad dada por la norma que estemos utilizando es convexa.

\begin{proposition}
    Sea $ p \colon X \to [0,\infty) $ una función con la propiedad de que para todo $ x \in X $ y para todo $ \lambda \in \R $ se tiene que $ p(\lambda x) = |\lambda| p(x) $. Dicha función satisface la desigualdad triangular si y solo si la bola  $B_p \coloneq \{ x \in X \colon p(x) \leq 1 \} $ es compacto.
\end{proposition}
\begin{proof}
    Si $ p $ satisface la desigualdad triangular, dados $ x, y \in B_p $, $ p(x) \leq 1 $ y  $ p(y) \leq 1 $, y tenemos que para todo $ t \in [0, 1] $,
    \begin{equation}
        p(tx + (1-t) y) \leq p(tx) + p((1-t)y) = tp(x) + (1-t)p(y) \leq t + (1-t) = 1.
    \end{equation}
    Por tanto, $ tx + (1-t)y \in B_p $ y $ B_p $ es convexa.

    Supongamos ahora que $ B_p $ es convexa. Para todo $ x, y \in X $ con $ p(x), p(y) \neq 0 $, definimos $ t \in (0, 1) $ como
    \begin{equation}
        t \coloneq \frac{p(x)}{p(x) + p(y)}, \text{ donde } 1 - t = \frac{p(y)}{p(x) + p(y)}.
    \end{equation}
    Tenemos que $ \frac{x}{p(x) } $ y $ \frac{x}{p(x) } $ están contenidos en $ B_p $ ya que, tomando el primer caso como ejemplo, 
    \begin{equation}
        p(\frac{x}{p(x)}) = \frac{p(x)}{p(x)} = 1.
    \end{equation}
    Ahora bien, como $ B_p $ es convexa, 
    \begin{equation}
        1 \geq p\left(t \frac{x}{p(x)} + (1-t) \frac{y}{p(y)}\right) = p\left( \frac{x}{p(x)+p(y)} + \frac{y}{p(x)+p(y)} \right).
    \end{equation}
    Por tanto, despejando tenemos
    \begin{equation}
        p(x)+p(y) \geq p(x+y).
    \end{equation}
\end{proof}

Vista la importancia de las bolas mediante esta caracterización, vamos a fijar notación que usaremos a lo largo del trabajo. Dado un espacio normado $ (X, \| \cdot \|) $, por conveniencia a veces denotado simplemente $ X $, definimos la \textbf{bola cerrada} y la \textbf{esfera unidad} en $ X $, $ B(X) $ y $ S(X) $ respectivamente como
\begin{equation}
\begin{split}
    B(X) &\coloneq \{x \in X \colon \| x \| \leq 1 \}, \\
    S(X) &\coloneq \{x \in X \colon \| x \| = 1 \}.
\end{split}
\end{equation}
Vamos a fijar también la notación que para las normas $ L^p $ que serán recurrentes a lo largo del trabajo. Sea  $(X, \mathcal{A}, \mu) $ un espacio de medida, y sea $f: X \to \R$ (o$\C$) una función medible. Para $1 \leq p < \infty$, la \textbf{norma $L^p$} de $f$ se define como:
\begin{equation}
    \|f\|_p \coloneq \left( \int_X |f(x)|^p \, d\mu(x) \right)^{1/p}.
\end{equation}
Para $p = \infty$, la \textbf{norma $L^\infty$} se define como:
\begin{equation}
\|f\|_\infty \coloneq \inf \{ M \geq 0 : |f(x)| \leq M \text{ para casi todo } x \in X \}.
\end{equation}
El espacio $L^p(X, \mu)$ se define como el conjunto de funciones $f$ medibles tales que $\|f\|_p < \infty$.

Para ciertos espacios (el Euclídeo por poner un ejemplo) siempre que dos puntos en la esfera unidad no sean el mismo, se tiene que la desigualdad triangular que relaciona sus normas es estricta. 

\begin{definition}[Espacio estrictamente convexo]
    Un espacio normado $ (X, \| \cdot \|) $ es {\bf estrictamente convexo} si para todo par de puntos $ x $ e $ y $ en la esfera unidad $ S(X) $ tales que el punto medio del segmento que los une esta también en la esfera unidad, i.e. $\| \frac{x+y}{2}\| $, se tiene $ x = y $.
\end{definition}

\begin{proposition} \label{prop:3}
    Para cualquier $\varepsilon \in (0,2]$, existe un $\delta \in (0,1)$ tal que, para todo $x, y$ en el círculo unitario con $\|x - y\|_2 > \varepsilon$, se tiene que
    \begin{equation}
        \left\| \frac{x + y}{2} \right\|_2 < \delta.
    \end{equation}
\end{proposition}
\begin{proof}
    Recordemos la generalización del teorema de Pitágoras para triángulos no rectángulos dada por la Ley de los Cosenos. Esta nos dice que, si $ a, b, c $ son los lados de un triangulo, y $ \gamma $ es el ángulo opuesto al lado $ c $, entonces
    \begin{equation}
        c^2 = a^2 + b^2 - 2 a b \cos(\gamma).
    \end{equation}
    Tomemos ahora los triángulos $\triangle Oxy$ y $\triangle Oxz$ de vertices el origen $ O $, $ x $ e $ y $ o $ z $ respectivamente. Sea $ \alpha $ el ángulo asociado al vértice $ x $. Aplicando la Ley de los Cosenos a estos triángulos tenemos que 
    \begin{equation}
        \cos(\alpha) = \frac{\|x\|_2^2 + \|x - y\|_2^2 - \|y\|_2^2}{2 \|x\|_2 \|x - y\|_2} = \frac{\|x\|_2^2 + \|x - z\|_2^2 - \|z\|_2^2}{2 \|x\|_2 \|x - z\|_2},
    \end{equation}
    lo cual, observando que $\|x\|_2 = \|y\|_2 = 1$ y que $x - z = \frac{x - y}{2}$, nos da
    \begin{equation}
        \frac{1}{2} \|x - y\|_2^2 = 1 + \frac{1}{4} \|x - y\|_2^2 - \|z\|_2^2.
    \end{equation}
    Por lo tanto, si $\|x - y\|_2 > \varepsilon$, entonces $\|z\|_2 < \delta$, siempre que
    \begin{equation}
        \delta > \sqrt{1 - \frac{1}{4} \varepsilon^2}.
    \end{equation}
\end{proof}

Tomando puntos no solo en la esfera, sino en toda la bola, podemos generalizar la proposición \refeq{prop:3}, introduciendo así los espacios uniformemente convexos.

\begin{definition}[Espacio uniformemente convexo]
    Un espacio normado $ (X, \| \cdot \|) $ es {\bf uniformemente convexo} si para todo $ \varepsilon \in (0, 2] $, existe un $ \delta \in (0, 1) $ tal que para todo par $ x, y \in B(X) $ con $ \| x - y \| < \varepsilon $ se tiene $ \| \frac{x+y}{2} \| < \delta $.
\end{definition}

La Proposición \ref{prop:3} afirma que $ \R^2 $ con la norma Euclídea es uniformemente convexo. El argumento se generaliza para $ \R^n $ para $ n \geq 2 $ o para $ \R $ de forma trivial. Por tanto, obtenemos el siguiente corolario.

\begin{corollary} \label{coro:1}
    Todo espacio uniformemente convexo es estrictamente convexo.
\end{corollary}

Veamos ahora que todo espacio dotado de un producto escalar, verificando la regla del paralelogramo, es uniformemente, y por tanto también estrictamente, convexo. En particular, el espacio $ L^2 $, será uniformemente convexo.

\begin{proposition} \label{prop:hilbert-convex}
    Sea $ (X, \langle \cdot, \cdot \rangle) $ un espacio dotado de un producto escalar con norma $ \|x\| \coloneq \sqrt{\langle x, x \rangle} $. Entonces el espacio $ (X, \|\cdot\|) $ es un espacio uniformemente convexo.
\end{proposition}
\begin{proof}
    La identidad del paralelogramo para normas inducidas por un producto escalar afirma
    \begin{equation}
        \|x + y\|^2 + \|x - y\|^2 = 2(\|x\|^2 + \|y\|^2).
    \end{equation}
    Sea \(x, y \in X\) con \(\|x\| = \|y\| = 1\) y \(x \neq y\). Definimos \(z = \frac{x + y}{2}\).  
    Entonces
    \begin{equation}
    \|z\|^2 = \left\|\frac{x + y}{2}\right\|^2 = \frac{1}{4} \|x + y\|^2.
    \end{equation}
    Aplicando la identidad del paralelogramo tenemos
    \begin{equation}
    \|x + y\|^2 = 2(\|x\|^2 + \|y\|^2) - \|x - y\|^2 = 4 - \|x - y\|^2.
    \end{equation}
    Por lo tanto,
    \begin{equation}
    \|z\|^2 = \frac{1}{4}(4 - \|x - y\|^2) = 1 - \frac{1}{4} \|x - y\|^2.
    \end{equation}
    Como \(x \neq y\), tenemos que \(\|x - y\| > 0\), lo cual implica
    \begin{equation}
    \|z\| < 1.
    \end{equation}
\end{proof}

Para ver que la implicación del Corolario \ref{coro:1} no funciona en sentido contrario, el siguiente ejemplo presenta una norma estrictamente convexa, pero no uniformemente convexa, construyéndose a partir de una combinación de normas en espacios de sucesiones.

\begin{example}[Ejemplo 7, \cite{babb}]
    Consideremos la norma en el espacio de sucesiones convergentes $ \ell^1 $ dada por la suma de las normas $ \ell^1 $ y $ \ell^2 $, es decir, dada una sucesión $ x \in \ell^1 $,
    \begin{equation}
        \| x \| \coloneq \| x \|_{\ell^1} + \| x \|_{\ell^2}.
    \end{equation}
    Gracias a la Proposición \ref{prop:hilbert-convex} sabemos que para todo $ x \neq y \in S(\ell^2) $,
    \begin{equation}
        \|x+y\|_{\ell^2} < \|x\|_{\ell^2} + \|y\|_{\ell^2}.
    \end{equation}
    Por tanto, haciendo también uso de la desigualdad triangular estándar en $ \ell^1 $ tenemos
    \begin{equation}
        \|x+y\| = \|x+y\|_{\ell^1} + \|x+y\|_{\ell^2} < \left( \|x\|_{\ell^1} + \|y\|_{\ell^1} \right) + \left( \|x\|_{\ell^2} + \|y\|_{\ell^2} \right) = \|x\| + \|y\| = 2.
    \end{equation}
    Por tanto, $ \ell_1 $ es estrictamente convexo. Sin embargo, definamos ahora las sucesiones
    \begin{align}
        x_{N, k} &=
        \begin{cases}
            1 & \text{si } k \leq N, \\
            0 & \text{en caso contrario},
        \end{cases}
        &
        y_{N, k} &=
        \begin{cases}
            1 & \text{si } N < k \leq 2N, \\
            0 & \text{en caso contrario}.
        \end{cases}
    \end{align}
    Por un lado tenemos que $ \|x\| = \|y\| = N + \sqrt{N} $ y que $ \|x_N - y_N\| = 2N + \sqrt{2N} $, ya que
    \begin{equation}
        x_{N, k} - y_N =
        \begin{cases}
            1 & \text{si } k \leq N, \\
            -1 & \text{si } N < k \leq 2N, \\
            0 & \text{si } k \geq 2N.
        \end{cases}
    \end{equation}
    Pero tenemos también que
    \begin{equation}
        \left\| \frac{x_N+y_N}{2} \right\| = \frac{2N}{2} + \frac{\sqrt{2N}}{2}.
    \end{equation}
    Dividiendo entre $ N + \sqrt{N} $ podemos hacer $ \left\| \frac{x_N+y_N}{2 (N + \sqrt{N})} \right\| $ tan cercano como queramos a $ 1 $, mientras que siempre tendremos $ \left\| \frac{x_N+y_N}{(N + \sqrt{N})} \right\| \geq \sqrt{2}$. Por tanto, $ (\ell^1, \| \cdot \|) $ no es uniformemente convexo.
\end{example}