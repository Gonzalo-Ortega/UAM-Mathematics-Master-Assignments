\documentclass[12pt, twoside]{article}

\usepackage[a4paper,bindingoffset=3mm,bottom=35mm]{geometry}
\usepackage[colorlinks=true, linkcolor=blue, citecolor=blue, pdftex]{hyperref}
\usepackage[pdftex]{graphicx}  
\usepackage[spanish]{babel}
\usepackage[utf8]{inputenc}
\usepackage{fullpage}

\usepackage{amsmath, amssymb, amsthm}
\usepackage{tikz-cd}
\usepackage{thm-restate}
\usepackage{mathtools}
\usepackage{fullpage}
\usepackage{quotchap}


\mathtoolsset{showonlyrefs}

\newtheorem{theorem}{Teorema}[section]
\newtheorem{proposition}[theorem]{Proposición}
\newtheorem{lemma}[theorem]{Lema}
\newtheorem{corollary}[theorem]{Corolario}
\newtheorem{fact}[theorem]{Hecho}

\theoremstyle{definition}
\newtheorem{example}[theorem]{Example}
\newtheorem{definition}[theorem]{Definition}
\newtheorem{remark}[theorem]{Remark}

\renewcommand{\qedsymbol}{$\blacksquare$}
\newcommand{\authorname}{Gonzalo Ortega Carpintero}
\newcommand{\institution}{Universidad Autónoma de Madrid}
\newcommand{\projecttitle}{Convexidad y Desigualdades en Espacios Normados}

\newcommand{\dgmp}{\operatorname{Dgm}_p}
\newcommand{\dgmi}{\operatorname{Dgm}_\infty}
\newcommand{\costp}{\operatorname{cost}_p}
\newcommand{\costi}{\operatorname{cost}_\infty}
\newcommand{\wdp}{\omega_p}
\newcommand{\wdi}{\omega_\infty}
\newcommand{\twdp}{\tilde \omega_p}
\newcommand{\twdi}{\tilde \omega_\infty}
\newcommand{\p}{\mathcal P}
\newcommand{\B}{\mathcal B}
\newcommand{\T}{T_\#}
\newcommand{\N}{\mathbb N}
\newcommand{\Z}{\mathbb Z}
\newcommand{\R}{\mathbb R}
\newcommand{\C}{\mathbb C}
\newcommand{\e}{\varepsilon}
\newcommand{\upr}{\mathbb{R}_<^2}

\newcommand{\barc}{\operatorname{Bar}}
\newcommand{\im}{\operatorname{im}}

\newcommand{\di}{d_{\operatorname{int}}}
\newcommand{\db}{d_{\operatorname{bot}}}
\newcommand{\dhf}{d_{\operatorname{H}}}
\newcommand{\dgh}{d_{\operatorname{GH}}}
\newcommand{\st}{\operatorname{St}}

\newcommand{\cf}{\check{\mathcal{C}}}
\newcommand{\rf}{\mathcal{R}}
\newcommand{\n}{\mathcal{N}}

% Custom header/footer configuration for article documents
\usepackage{fancyhdr}

\pagestyle{fancy}
\fancyhf{} % Clear all headers/footers

% Article version (no chapters)
\renewcommand{\sectionmark}[1]{\markright{Sección \thesection.\ #1}}
\renewcommand{\subsectionmark}[1]{\markright{Hola \thesubsection.\ #1}}

% Headers (adjust based on single/double-sided needs)
\fancyhead[LO]{\nouppercase\rightmark}
\fancyhead[RO]{\thepage}
\fancyhead[RE]{\projecttitle}
\fancyhead[LE]{\thepage}

% Footer (optional)
\fancyfoot[C]{} % Empty footer

% Header spacing
\setlength{\headheight}{14.5pt}
\setlength{\headsep}{30pt}


\begin{document}
\begin{titlepage}
    \centering
    %{\includegraphics[width=0.2\textwidth]{logo}\par}
    \vspace{1cm}
    {\bfseries\LARGE \institution \par}
    \vspace{1cm}
    {\scshape\Large Curso Avanzado de Análisis \par}
    \vspace{3cm}
    {\scshape\Huge \projecttitle \par}
    \vspace{3cm}
    {\itshape\Large   
    2024-2025. \par}
    \vfill
    {\Large Autor: \par}
    {\Large \authorname \par}
    \vfill
    {\Large Mayo 2025 \par}
\end{titlepage}



\setlength{\parskip}{0.75em}
\renewcommand{\baselinestretch}{1.25}

\subsection*{Resumen}
A lo largo de este trabajo se da una breve introducción a los espacios convexos relacionándolos con la desigualdad triangular de los espacios normados. Se define la desigualdad uniforme definida por primera vez en 1936 por James A. Clarkson \cite{clarkson}. Así pues, se introducen las desigualdades de Clarkson y de Hanner, sustitutos débiles de la identidad del paralelogramo que son validos en $ L^p $ para todo $ p \in (1, \infty) $, en lugar de únicamente para $ p = 2 $. Siguiendo los pasos de Clarkson, usamos sus desigualdades para probar la convexidad uniforme de dichos espacios $ L^p $. Finalmente, concluimos comparando ambas desigualdades para observar como las desigualdades de Hanner dan una mejor acotación que las de Clarkson.

\subsection*{Palabras clave}
Convexidad, convexidad uniforme,

desigualdades de Clarkson, desigualdades de Hanner.

\setcounter{page}{1}
\tableofcontents
\newpage

\section{Introduction}

Transport maps were introduced in 1781 by Gaspard Monge to represent the idea of moving earth from one place into an other \cite{Figalli}[1.1 Historical overview]. In this original formulation of the optimal transport problem, it was enough to consider $ \mathbb R^3 $ as the ambient space, using the Euclidean distance as the cost function of moving mass between two points.

In the 30's, Leonid Kantorovich reformulated the problem to describe the optimization process of supply and demand distributions of diverse problems. The mass could be divided between different origin and destinations, making it possible to interpret the problem as the way to measure the cost of transforming one probability distribution into an other. In this thesis, will introduce the $p$-Wasserstein distance as a metric on the probability measures with finite $p$-moment space. When $ p = 1 $, the distance will represent the metric introduced in the Kantorovich optimal transport problem, also used named Earth Mover's distance, used for machine learning algorithms and computer vision problems \cite{earth}. When $p = \infty $ it is named the bottleneck distance, and will be the main them of study of this thesis.

In topological data analysis, diagrams arise to represent the 
persistence of the homology groups of a data set through time. Those diagrams are named persistence diagrams, and those homology groups, persistence homology groups. We will introduce an analogous $p$-Wasserstein distance in the space of persistence diagrams and prove that there exists an isometric embedding  from a separable metric space into the space of persistence diagrams with the Wasserstein distance.
\newpage
\section{Espacios convexos}

La principal herramienta para trabajar en espacios métricos, y en particular en los espacios normados \cite{babb}, es la desigualdad triangular 
\begin{equation}
    \| x + y \| \leq \| x \| + \| y \|.
\end{equation}

\begin{definition}[Conjunto convexo]
    Un conjunto $ C \subseteq X $ es {\bf convexo} si para todo par de puntos $ x, y \in C $, $ t \in [0, 1] $, se tiene $ tx + (1-t)y \in C $.
\end{definition}

\begin{proposition}
    Sea $ p \colon X \to [0,\infty) $ una función con la propiedad de que para todo $ x \in X $ y para todo $ \lambda \in \R $ se tiene que $ p(\lambda x) = |\lambda| p(x) $. Dicha función satisface la desigualdad triangular si y solo si la bola  $B_p \coloneq \{ x \in X \colon p(x) \leq 1 \} $ es compacto.
\end{proposition}
\begin{proof}
    Si $ p $ satisface la desigualdad triangular, dados $ x, y \in B_p $, $ p(x) \leq 1 $ y  $ p(y) \leq 1 $, y tenemos que para todo $ t \in [0, 1] $,
    \begin{equation}
        p(tx + (1-t) y) \leq p(tx) + p((1-t)y) = tp(x) + (1-t)p(y) \leq t + (1-t) = 1.
    \end{equation}
    Por tanto, $ tx + (1-t)y \in B_p $ y $ B_p $ es convexa.

    Supongamos ahora que $ B_p $ es convexa. Para todo $ x, y \in X $ con $ p(x), p(y) \neq 0 $, definimos $ t \in (0, 1) $ como
    \begin{equation}
        t \coloneq \frac{p(x)}{p(x) + p(y)}, \text{ donde } 1 - t = \frac{p(y)}{p(x) + p(y)}.
    \end{equation}
    Tenemos que $ \frac{x}{p(x) } $ y $ \frac{x}{p(x) } $ están contenidos en $ B_p $ ya que, tomando el primer caso como ejemplo, 
    \begin{equation}
        p(\frac{x}{p(x)}) = \frac{p(x)}{p(x)} = 1.
    \end{equation}
    Ahora bien, como $ B_p $ es convexa, 
    \begin{equation}
        1 \geq p\left(t \frac{x}{p(x)} + (1-t) \frac{y}{p(y)}\right) = p\left( \frac{x}{p(x)+p(y)} + \frac{y}{p(x)+p(y)} \right).
    \end{equation}
    Por tanto, despejando tenemos
    \begin{equation}
        p(x)+p(y) \geq p(x+y).
    \end{equation}
\end{proof}

Dado un espacio normado $ (X, \| \cdot \|) $, por conveniencia a veces denotado simplemente $ X $, definimos la bola cerrada y la esfera unidad en $ X $, $ B(X) $ y $ S(X) $ respectivamente como
\begin{equation}
\begin{split}
    B(X) &\coloneq \{x \in X \colon \| x \| \leq 1 \}, \\
    S(X) &\coloneq \{x \in X \colon \| x \| = 1 \}.
\end{split}
\end{equation}

\begin{definition}[Espacio estrictamente convexo]
    Un espacio normado $ (X, \| \cdot \|) $ es {\bf estrictamente convexo} si para todo par de puntos $ x $ e $ y $ en la esfera unidad $ S(X) $ tales que el punto medio del segmento que los une esta también en la esfera unidad, i.e. $\| \frac{x+y}{2}\| $, se tiene $ x = y $.
\end{definition}

\begin{definition}[Espacio uniformemente convexo]
    Un espacio normado $ (X, \| \cdot \|) $ es {\bf uniformemente convexo} si para todo $ \varepsilon \in (0, 2] $, existe un $ \delta \in (0, 1) $ tal que para todo par $ x, y \in B(X) $ con $ \| x - y \| < \varepsilon $ se tiene $ \| \frac{x+y}{2} \| < \delta $.
\end{definition}

\begin{corollary} \label{coro:1}
    Todo espacio uniformemente convexo es estrictamente convexo.
\end{corollary}

\begin{proposition} \label{prop:hilbert-convex}
    Sea $ (X, \langle \cdot, \cdot \rangle) $ un espacio dotado de un producto escalar con norma $ \|x\| \coloneq \sqrt{\langle x, x \rangle} $. Entonces el espacio $ (X, \|\cdot\|) $ es un espacio uniformemente, y por tanto también estrictamente, convexo. 
\end{proposition}

Para ver que la implicación del Corolario \ref{coro:1} no funciona en sentido contrario, el siguiente ejemplo presenta una norma estrictamente convexa, pero no uniformemente convexa, construyéndose a partir de una combinación de normas en espacios de sucesiones.

\begin{example}[Ejemplo 7, \cite{babb}]
    Consideremos la norma en el espacio de sucesiones convergentes $ \ell^1 $ dada por la suma de las normas $ \ell^1 $ y $ \ell^2 $, es decir, dada una sucesión $ x \in \ell^1 $,
    \begin{equation}
        \| x \| \coloneq \| x \|_{\ell^1} + \| x \|_{\ell^2}.
    \end{equation}
    Gracias a la Proposición \ref{prop:hilbert-convex} sabemos que para todo $ x \neq y \in S(\ell^2) $,
    \begin{equation}
        \|x+y\|_{\ell^2} < \|x\|_{\ell^2} + \|y\|_{\ell^2}.
    \end{equation}
    Por tanto, haciendo también uso de la desigualdad triangular estándar en $ \ell^1 $ tenemos
    \begin{equation}
        \|x+y\| = \|x+y\|_{\ell^1} + \|x+y\|_{\ell^2} < \left( \|x\|_{\ell^1} + \|y\|_{\ell^1} \right) + \left( \|x\|_{\ell^2} + \|y\|_{\ell^2} \right) = \|x\| + \|y\| = 2.
    \end{equation}
    Por tanto, $ \ell_1 $ es estrictamente convexo. Sin embargo, definamos ahora las sucesiones
    \begin{align}
        x_{N, k} &=
        \begin{cases}
            1 & \text{si } k \leq N, \\
            0 & \text{en caso contrario},
        \end{cases}
        &
        y_{N, k} &=
        \begin{cases}
            1 & \text{si } N < k \leq 2N, \\
            0 & \text{en caso contrario}.
        \end{cases}
    \end{align}
    Por un lado tenemos que $ \|x\| = \|y\| = N + \sqrt{N} $ y que $ \|x_N - y_N\| = 2N + \sqrt{2N} $, ya que
    \begin{equation}
        x_{N, k} - y_N =
        \begin{cases}
            1 & \text{si } k \leq N, \\
            -1 & \text{si } N < k \leq 2N, \\
            0 & \text{si } k \geq 2N.
        \end{cases}
    \end{equation}
    Pero tenemos también que
    \begin{equation}
        \left\| \frac{x_N+y_N}{2} \right\| = \frac{2N}{2} + \frac{\sqrt{2N}}{2}.
    \end{equation}
    Dividiendo entre $ N + \sqrt{N} $ podemos hacer $ \left\| \frac{x_N+y_N}{2 (N + \sqrt{N})} \right\| $ tan cercano como queramos a $ 1 $, mientras que siempre tendremos $ \left\| \frac{x_N+y_N}{(N + \sqrt{N})} \right\| \geq \sqrt{2}$. Por tanto, $ (\ell^1, \| \cdot \|) $ no es uniformemente convexo.
\end{example}
\newpage


\section{Desigualdades de Clarkson}

Los espacios de funciones $ L^p $ no son por lo general espacios de Hilbert ya que, salvo para $ p = 2 $, no cumplen la regla del paralelogramo
$$
    \| x + y \|^2 + \| x - y \|^2 = 2 \left( \| x \|^2 + \| y\|^2 \right)
$$
En 1936, el matemático americano James A. Clarkson generalizó la regla del paralelogramo para hacerla válida para todo $ p \geq 1 $ en forma de dos nuevas desigualdades, una para $p \geq 2 $ y otra para $ 1 \leq p \leq 2 $, \cite{clarkson}. Si bien esto no permite definir un producto escalar sobre cualquier $ L^p $, si que permite comprobar que, como veremos, $ L^p $ es uniformemente convexo para $ p \geq 1 $.

\begin{theorem}[Primera desigualdad de Clarkson] \label{thm:clarkson-1}
    Para $ p \geq 2 $, dadas $ f, g \in L^p $, se verifica la desigualdad
    \begin{align} \label{eq:clarkson-1}
        \left\| f+g \right\|_p^p + \left\| f-g \right\|_p^p \leq 2^{p-1} \left( \left\|f\right\|_p^p + \left\|g\right\|_p^p \right).
    \end{align}
\end{theorem}

\begin{theorem}[Segunda desigualdad de Clarkson]
    Sea $ 1 < p < 2 $, y sea $ q = \frac{p-1}{p}$. Para $ f, g \in L^p $ se tiene
    \begin{align} \label{eq:clarkson-2}
        \left\| f + g \right\|_p^q + \left\| f - g \right\|_p^q \leq 2^p \left(\|f \|_p^p + \|g\|_p^p \right)^{\frac{q}{p}}.
    \end{align}
\end{theorem}

Para desarrollar la demostración de ambas desigualdades hemos seguido la estructura de \cite{hewitt}. Empecemos probando un par de lemas auxiliares.

\begin{lemma} \label{lema:clarkson-1}
    Sea $ x \in [0, 1] $ y $ p \geq 2 $. Se verifica la desigualdad
    \begin{align}
        \left( \frac{1+x}{2} \right)^p + \left( \frac{1-x}{2} \right)^p \leq \frac{1}{2} (1 + x^p).
    \end{align}
\end{lemma}
\begin{proof}
    Definiendo la función
    \begin{align}
        F(x) \coloneq \left( \frac{1+x}{2} \right)^p + \left( \frac{1-x}{2} \right)^p - \frac{1}{2} (1 + x^p),
    \end{align}
    sería suficiente probar que $ F(x) \geq 0 $ para todo $ x \in [0, 1] $. Para $ x = 0 $, como $ p \geq 2 $, se tiene 
    \begin{align}
        F(0) = \frac{1}{2^p} + \frac{1}{2^p} - \frac{1}{2} \leq 0.
    \end{align}
    Para $ 0 < x \leq 1 $ definimos
    \begin{align}
        \Phi(x) \coloneq \frac{2^p}{x^p} F(x) = \left( \frac{1}{x} + 1 \right)^p + \left( \frac{1}{x} - 1 \right)^p - 2^{p-1} \left( \frac{1}{x^p} + 1 \right).
    \end{align}
    Para $ x = 1 $, $ \phi(1) = 0 $, luego veamos que $ \phi $ es creciente en el intervalo $ (0, 1) $. La derivada de $ \phi $ es
    \begin{align}
        \Phi'(x) = -\frac{p}{x^{p+1}} \left( (1+x)^{p-1} + (1-x)^{p-1} - 2^{p-1} \right).
    \end{align}
    Definiendo ahora la función $ \Psi(x) $ como la parte entre paréntesis de $ Phi$, y calculando su derivada tenemos
    \begin{align}
        \Psi(x) &\coloneq \left( (1+x)^{p-1} + (1-x)^{p-1} - 2^{p-1} \right), \\
        \Psi'(x) &= (p-1)(1+x)^{p-2} - (p-1)(1-x)^{p-2}.
    \end{align}
    Luego $ \Psi'(x) \geq 0 $ para $ x \in (0, 1) $. Como $ \Psi(1) = 0 $, por el teorema del valor medio $ \Psi(x) \leq 0 $ para $ x \in (0, 1) $. Por tanto $ \Phi'(x) \geq 0 $ para $ x \in (0, 1) $ y como $ \Phi(1) = 0 $, $ \Phi(x) $ es no positiva para $ x \in (0, 1) $. Esto implica finalmente que $ F(x) \leq 0 $ para todo $ x \in (0, 1) $.
\end{proof}

\begin{lemma} \label{lema:clarkson-2}
    Sean $ z, w \in \C $ dos números complejos, donde si $ z = a + bi $, denotamos su módulo complejo como $|z| = \sqrt{a^2 + b^2} $. Dado $ p \geq 2 $ se verifica la desigualdad
    \begin{align}
        \left| \frac{1}{2}(z+w)\right|^p + \left| \frac{1}{2}(z-w)\right|^p \leq \frac{1}{2} |z|^p + \frac{1}{2} |w|^p.
    \end{align}
\end{lemma}
\begin{proof}
    Para el caso $ w = 0 $ es inmediato que se verifica la desigualdad ya que se tendría
    $$
        \left| \frac{z}{2}\right|^p + \left| \frac{z}{2}\right|^p = 2\left|\frac{z}{2}\right|^p = \frac{2}{2^p}\left|z\right|^p = \frac{1}{2^{p-1}}\left|z\right|^p \leq \frac{1}{2}\left|z\right|^p.
    $$
    Por tanto, y gracias a la simetría de los dos sumandos del lado derecho, podemos asumir sin pérdida de generalidad $ |z| \geq |w| > 0 $. Es decir, la desigualdad que queremos probar equivale, al dividir a ambos lados entre $ |z|^p $, a
    $$
        \left| \frac{1}{2}(1+\frac{w}{z})\right|^p + \left| \frac{1}{2}(1-\frac{w}{z})\right|^p \leq \frac{1}{2}\left(1 + \left|\frac{w}{z} \right|^p \right).
    $$
    Por tanto, tomando exponenciales, para $ 0 < r \leq 1 $ y $ 0 \leq \theta \leq 2 \pi $ tenemos
    $$
        \left| \frac{1+r \exp (i \theta)}{2} \right|^p + \left| \frac{1-r \exp (i \theta)}{2} \right|^p \leq \frac{1}{2}\left(1 + \frac{1+r \exp (i \theta)}{2}^p \right).
    $$
    Para $ \theta = 0 $ la desigualdad se reduce a la probada en el Lema \ref{lema:clarkson-1}. Veamos por tanto que dado un $ r $ fijo, se tiene un máximo en $ \theta = 0 $. Por la simetría del lado derecho de nuevo, podemos asumir que $ 0 \leq \theta \leq \frac{\pi}{2} $. Queremos por tanto probar que la función $g$ definida por
    $$
        g(\theta) = |1 + re^{i\theta}|^p + |1 - re^{i\theta}|^p
    $$
    tiene un máximo en el intervalo $[0, \frac{\pi}{2}] $ en el punto $ \theta = 0 $. Desarrollando la fórmula de Euler, $ e^{i\theta} = \cos(\theta) + i\sen(\theta)$,  y los módulos complejos tenemos
    \begin{align}
        g(\theta) &= \left|\sqrt{(1 + r \cos(\theta))^2 + (r\sen(\theta))^2} \right|^p + \left|\sqrt{(1 - r \cos(\theta))^2 + (- r\sen(\theta))^2} \right|^p \\
        &= (1 + r^2 + 2r\cos(\theta))^\frac{p}{2} + (1 + r^2 - 2r\cos(\theta))^\frac{p}{2}
    \end{align}
    Tomamos ahora la derivada $g'$ de $ g $ respecto a $ \theta $ y observamos
    \begin{align}
        g'(\theta) = &\frac{p}{2}(1 + r^2 + 2r \cos(\theta))^{\frac{p}{2}-1} (-2r \sen(\theta)) + \frac{p}{2}(1 + r^2 - 2r \cos(\theta))^{\frac{p}{2}-1} (2r \sen(\theta)) \\
        = &-pr \sen(\theta) \left((1+r^2 + 2r\cos(\theta))^{\frac{p}{2}-1} - (1+r^2 - 2r\cos(\theta))^{\frac{p}{2}-1}\right).
    \end{align}
    Como $ p \geq 2 $ entonces $g'(\theta) \leq 0 $. Es decir, la derivada de $ g $ no es creciente en todo $ \theta \in \left[0, \frac{\pi}{2}\right] $ y por tanto tiene un máximo en $ \theta = 0 $.
\end{proof}

\begin{proof}[Demostración del Teorema \ref{thm:clarkson-1}]
    Podemos asumir que $ f $ y $ g $ toman valores complejos y que están definidas en casi todo punto. Por tanto, para todo $ x \in X $, tal que $ f(x) $ y $ g(x) $ estén definidas, por el Lemma \ref{lema:clarkson-2} tenemos
    \begin{align}
        \left| z + w \right|^p + \left| z - w \right|^p \leq 2^{p-1} \left( |z|^p + |w|^p \right).
    \end{align}
    Basta con integrar a ambos lados respecto a $ X $ para la desigualdad \eqref{eq:clarkson-1}.
\end{proof}

\begin{corollary}
    Para $ p > 1 $, el espacio $ L_p $ es uniformemente convexo.
\end{corollary}
\begin{proof}
    Sean $ f, g \in L^p $ con $ \|f\| = \|g\| = 1 $. Para $ p \geq 2 $, haciendo uso de \eqref{eq:clarkson-1} tenemos
    \begin{align}
        \| f + g \|_p^p + \| f - g \|_p^p \leq 2^{p-1}(1 + 1) = 2^p.
    \end{align}
    Ahora, si se tiene $\|f-g\| < \varepsilon \in (0, 2] $, dividiendo entre $ 2^p $ a ambos lados y despejando adecuadamente se sigue
    \begin{align}
        \left\| \frac{f + g}{2} \right\|_p \leq \left( 1 - \left\| \frac{f - g}{2} \right\|_p^p\right)^{\frac{1}{p}} \leq \left( 1 - \left( \frac{\varepsilon}{2} \right)^p\right)^{\frac{1}{p}} \eqcolon \delta,
    \end{align}
    donde $ \delta \in (0, 1) $ y se verifica la convexidad uniforme para $ p \geq 2 $. Para $ 1 < p \leq 2 $ usamos la desigualdad \eqref{eq:clarkson-2} para obtener
    \begin{align}
        \left\| f + g \right\|_p^q + \left\| f - g \right\|_p^q \leq 2 \left(1 + 1 \right)^{\frac{q}{p}} = 2^{\frac{q}{p}+1} = 2^q.
    \end{align}
    Análogamente al procedimiento anterior, ahora con $ \delta \coloneq  \left( 1 - \left( \frac{\varepsilon}{2} \right)^q\right)^{\frac{1}{q}} $ vemos que se verifica la convexidad uniforme también para $ 1 < p \leq 2 $.
\end{proof}

\subsection{Desigualdades de Hanner}
En 1955, el matemático sueco Olof Hanner simplificó la prueba de la convexidad uniforme dada por Clarkson años antes \cite{hanner}. Para ello, introdujo dos nuevas desigualdades que también generalizan la regla del paralelogramo.

\begin{theorem}[Desigualdades de Hanner, Teorema 2.5, \cite{lieb}]
    Sean $ f $ y $ g $ dos funciones en $L^p$. Para $ p \geq 2 $, se verifica
    \begin{equation} \label{eq:hanner}
        \left\| f+g \right\|_p^p + \left\| f-g \right\|_p^p \leq \left( \|f \|_p + \|g\|_p \right)^p + \left| \|f \|_p - \|g\|_p \right|^p.
    \end{equation}
    Para $ 1 < p \leq 2 $, la desigualdad se invierte.
\end{theorem}
Antes de nada, notar que para $ p = 2 $, el lado de la derecha de \eqref{eq:hanner} es
\begin{align}
    \left( \|f \|_2 + \|g\|_2 \right)^2 + \left| \|f \|_2 - \|g\|_2 \right|^2 &= \|f \|_2^2 + \|g\|_2^2 + 2 \|f \|_2 \|g\|_2 + \|f \|_2^2 + \|g\|_2^2 - 2 \|f \|_2 \|g\|_2 \\
    &= 2 \left(\|f \|_2^2 + \|g\|_2^2\right),
\end{align}
luego efectivamente se convierte en la ley del paralelogramo, la cual se verifica para $ p = 2 $.
Vamos a dar una prueba siguiendo la dada en \cite{lieb}[Teorema 2.5].

\begin{proof}
Sin pérdida de generalidad, podemos asumir $\|f\|_p = 1$ y definimos $M \coloneq \|g\|_p \leq 1$. Para $0 \leq x \leq 1$, consideramos las funciones:
\begin{align}
    \Phi(x) &\coloneq (1 + x)^{p-1} + (1 - x)^{p-1}, \\
    \Psi(x) &\coloneq \left[(1 + x)^{p-1} - (1 - x)^{p-1}\right] x^{1-p}, \\
    F_M(x) &\coloneq \Phi(x) + \Psi(x) M^p.
\end{align}
Calculando la derivada de $ F_M $, 
\begin{equation}
    F_M'(x) = \Phi'(x) + \Psi'(x) M^p = (p-1)\left[(1 + x)^{p-2} - (1 - x)^{p-2}\right]\left(1 - \left(\frac{M}{x}\right)^p\right)
\end{equation}
observamos que $F_M'(x) = 0$ solo cuando $x = M$, por lo que para $p < 2$, $F_M(x)$ tiene un máximo en $x = M$, y para $p > 2$, $F_M(x)$ tiene un mínimo en $x = M$. Además, en ambos casos se cumple:
\begin{equation}
    F_M(M) = (1 + M)^p + (1 - M)^p.
\end{equation}
Para $p < 2$ tenemos $\Psi(x) \leq \Phi(x)$, y para $p > 2$ tenemos $\Psi(x) \geq \Phi(x)$. Esto implica que para todo $0 \leq x \leq 1$ y $A, B \geq 0$:
\begin{equation}
    \Phi(x)|A|^p + \Psi(x)|B|^p \leq |A + B|^p + |A - B|^p \quad \text{para } p < 2
\end{equation}
y la desigualdad inversa para $p > 2$. La igualdad se alcanza cuando $x = B/A \leq 1$. Esta desigualdad se puede extender a números complejos $A$ y $B$. Para ver esto, consideramos $A = a$ y $B = b e^{i\theta}$ con $a, b > 0$. Tenemos que
\begin{equation}
    (a^2 + b^2 + 2ab \cos \theta)^{p/2} + (a^2 + b^2 - 2ab \cos \theta)^{p/2}
\end{equation}
alcanza su mínimo en $\theta = 0$ para $p < 2$ (y su máximo para $p > 2$), debido a que $x \mapsto x^{r}$ es cóncava para $0 < r < 1$ y convexa para $r > 1$. Para completar la demostración, integramos ahora la desigualdad puntual
\begin{equation}
    |f + g|^p + |f - g|^p \geq \Phi(x)|f|^p + \Psi(x)|g|^p \quad \text{para } p < 2
\end{equation}
(con la desigualdad inversa para $p > 2$). Tomando $x = M = \|g\|_p$ (pues asumimos $\|f\|_p = 1$) obtenemos
\begin{equation}
    \|f + g\|_p^p + \|f - g\|_p^p \geq (1 + M)^p + (1 - M)^p = (\|f\|_p + \|g\|_p)^p + |\|f\|_p - \|g\|_p|^p
\end{equation}
para $1 \leq p \leq 2$, y la desigualdad inversa para $p \geq 2$, lo que completa la demostración.
\end{proof}

\subsection{Una pequeña comparación}

Las desigualdades de Hanner dan una mejor acotación de la convexidad uniforme de $ L^p $. Veámoslo para el caso $ 2 \leq p < \infty $. La desigualdad de Clarkson y la de Hanner afirman, respectivamente, 
\begin{align}
    \left\| f+g \right\|_p^p + \left\| f-g \right\|_p^p &\leq 2^{p-1} \left( \|f\|_p^p + \|g\|_p^p \right), \\
    \left\| f+g \right\|_p^p + \left\| f-g \right\|_p^p &\leq \left( \|f \|_p + \|g\|_p \right)^p + \left| \|f \|_p - \|g\|_p \right|^p.
\end{align}
Por tanto, basta tomar las funciones constantes $ f' = \|f \|_p $ y $ g' = \|g\|_p $ donde, naturalmente, $ f', g' \in L^p $ y
\begin{align}
    \left( \|f \|_p + \|g\|_p \right)^p = \left( f' + g' \right)^p = \left\| f'+g' \right\|_p^p, \\
    \left| \|f \|_p - \|g\|_p \right|^p = \left| f' - g' \right|^p = \left\| f'-g' \right\|_p^p.
\end{align}
Así pues, utilizando aplicando tanto Clarkson como Hanner, obtenemos la siguiente cadena de desigualdades, probando la mejor cota dada por Hanner:
\begin{align}
     \left\| f+g \right\|_p^p + \left\| f-g \right\|_p^p &\leq  \left( \|f \|_p + \|g\|_p \right)^p + \left| \|f \|_p - \|g\|_p \right|^p = \left\| f'+g' \right\|_p^p + \left\| f'-g' \right\|_p^p \\
     &\leq 2^{p-1} \left( \|f'\|_p^p + \|g'\|_p^p \right) = 2^{p-1} \left( \|f\|_p^p + \|g\|_p^p \right).
\end{align}
Nótese también que efectivamente las desigualdades no pueden ser estrictas ya que para el caso $ \|f\| = \|g\| = 1 $, tanto la desigualdad de Clarkson como la de Hanner dan la misma cota, $ 2^p $, la cual permite probar la convexidad uniforme de $ L^p $ haciendo uso de cualquiera de las desigualdades. La prueba usando Clarkson la hemos visto en el Corolario \ref{coro:clarkson}, y la prueba análoga, aunque dando una mejor cota se puede consultar en \cite{hanner}.
\newpage

\bibliographystyle{acm}
\bibliography{doc-configuration/biblio.bib}

\end{document}
