\section{Desigualdades de Clarkson}

Los espacios de funciones $ L^p $ no son por lo general espacios de Hilbert ya que, salvo para $ p = 2 $, no cumplen la regla del paralelogramo
\begin{equation}
    \| x + y \|^2 + \| x - y \|^2 = 2 \left( \| x \|^2 + \| y\|^2 \right)
\end{equation}
En 1936, el matemático americano James A. Clarkson generalizó la regla del paralelogramo para hacerla válida para todo $ p \geq 1 $ en forma de dos nuevas desigualdades, una para $p \geq 2 $ y otra para $ 1 \leq p \leq 2 $, \cite{clarkson}. Si bien esto no permite definir un producto escalar sobre cualquier $ L^p $, si que permite comprobar que, como veremos, $ L^p $ es uniformemente convexo para $ p \geq 1 $.

\begin{restatable}[Primera desigualdad de Clarkson]{theorem}{Clarksoni} \label{thm:clarkson-1}
    Para $ p \geq 2 $, dadas $ f, g \in L^p $, se verifica la desigualdad
    \begin{equation} \label{eq:clarkson-1}
        \left\| f+g \right\|_p^p + \left\| f-g \right\|_p^p \leq 2^{p-1} \left( \left\|f\right\|_p^p + \left\|g\right\|_p^p \right).
    \end{equation}
\end{restatable}

\begin{restatable}[Segunda desigualdad de Clarkson]{theorem}{Clarksonii} \label{thm:clarkson-2}
    Sea $ 1 < p < 2 $, y sea $ q = \frac{p}{p-1}$. Para $ f, g \in L^p $ se tiene
    \begin{equation} \label{eq:clarkson-2}
        \left\| f + g \right\|_p^q + \left\| f - g \right\|_p^q \leq 2 \left(\|f \|_p^p + \|g\|_p^p \right)^{\frac{q}{p}}.
    \end{equation}
\end{restatable}

Para desarrollar la demostración de ambas desigualdades hemos seguido la estructura de \cite{hewitt}. Empecemos probando un par de lemas auxiliares.

\begin{lemma} \label{lema:clarkson-1}
    Sea $ x \in [0, 1] $ y $ p \geq 2 $. Se verifica la desigualdad
    \begin{align}
        \left( \frac{1+x}{2} \right)^p + \left( \frac{1-x}{2} \right)^p \leq \frac{1}{2} (1 + x^p).
    \end{align}
\end{lemma}
\begin{proof}
    Definiendo la función
    \begin{align}
        F(x) \coloneq \left( \frac{1+x}{2} \right)^p + \left( \frac{1-x}{2} \right)^p - \frac{1}{2} (1 + x^p),
    \end{align}
    sería suficiente probar que $ F(x) \geq 0 $ para todo $ x \in [0, 1] $. Para $ x = 0 $, como $ p \geq 2 $, se tiene 
    \begin{equation}
        F(0) = \frac{1}{2^p} + \frac{1}{2^p} - \frac{1}{2} \leq 0.
    \end{equation}
    Para $ 0 < x \leq 1 $ definimos
    \begin{align}
        \Phi(x) \coloneq \frac{2^p}{x^p} F(x) = \left( \frac{1}{x} + 1 \right)^p + \left( \frac{1}{x} - 1 \right)^p - 2^{p-1} \left( \frac{1}{x^p} + 1 \right).
    \end{align}
    Para $ x = 1 $, $ \phi(1) = 0 $, luego veamos que $ \phi $ es creciente en el intervalo $ (0, 1) $. La derivada de $ \phi $ es
    \begin{equation}
        \Phi'(x) = -\frac{p}{x^{p+1}} \left( (1+x)^{p-1} + (1-x)^{p-1} - 2^{p-1} \right).
    \end{equation}
    Definiendo ahora la función $ \Psi(x) $ como la parte entre paréntesis de $ Phi$, y calculando su derivada tenemos
    \begin{equation}
    \begin{split}
        \Psi(x) &\coloneq \left( (1+x)^{p-1} + (1-x)^{p-1} - 2^{p-1} \right), \\
        \Psi'(x) &= (p-1)(1+x)^{p-2} - (p-1)(1-x)^{p-2}.
    \end{split}
    \end{equation}
    Luego $ \Psi'(x) \geq 0 $ para $ x \in (0, 1) $. Como $ \Psi(1) = 0 $, por el teorema del valor medio $ \Psi(x) \leq 0 $ para $ x \in (0, 1) $. Por tanto $ \Phi'(x) \geq 0 $ para $ x \in (0, 1) $ y como $ \Phi(1) = 0 $, $ \Phi(x) $ es no positiva para $ x \in (0, 1) $. Esto implica finalmente que $ F(x) \leq 0 $ para todo $ x \in (0, 1) $.
\end{proof}

\begin{lemma} \label{lema:clarkson-2}
    Sean $ z, w \in \C $ dos números complejos, donde si $ z = a + bi $, denotamos su módulo complejo como $|z| = \sqrt{a^2 + b^2} $. Dado $ p \geq 2 $ se verifica la desigualdad
    \begin{align}
        \left| \frac{1}{2}(z+w)\right|^p + \left| \frac{1}{2}(z-w)\right|^p \leq \frac{1}{2} |z|^p + \frac{1}{2} |w|^p.
    \end{align}
\end{lemma}
\begin{proof}
    Para el caso $ w = 0 $ es inmediato que se verifica la desigualdad ya que se tendría
    $$
        \left| \frac{z}{2}\right|^p + \left| \frac{z}{2}\right|^p = 2\left|\frac{z}{2}\right|^p = \frac{2}{2^p}\left|z\right|^p = \frac{1}{2^{p-1}}\left|z\right|^p \leq \frac{1}{2}\left|z\right|^p.
    $$
    Por tanto, y gracias a la simetría de los dos sumandos del lado derecho, podemos asumir sin pérdida de generalidad $ |z| \geq |w| > 0 $. Es decir, la desigualdad que queremos probar equivale, al dividir a ambos lados entre $ |z|^p $, a
    $$
        \left| \frac{1}{2}(1+\frac{w}{z})\right|^p + \left| \frac{1}{2}(1-\frac{w}{z})\right|^p \leq \frac{1}{2}\left(1 + \left|\frac{w}{z} \right|^p \right).
    $$
    Por tanto, tomando exponenciales, para $ 0 < r \leq 1 $ y $ 0 \leq \theta \leq 2 \pi $ tenemos
    $$
        \left| \frac{1+r \exp (i \theta)}{2} \right|^p + \left| \frac{1-r \exp (i \theta)}{2} \right|^p \leq \frac{1}{2}\left(1 + \frac{1+r \exp (i \theta)}{2}^p \right).
    $$
    Para $ \theta = 0 $ la desigualdad se reduce a la probada en el Lema \ref{lema:clarkson-1}. Veamos por tanto que dado un $ r $ fijo, se tiene un máximo en $ \theta = 0 $. Por la simetría del lado derecho de nuevo, podemos asumir que $ 0 \leq \theta \leq \frac{\pi}{2} $. Queremos por tanto probar que la función $g$ definida por
    $$
        g(\theta) = |1 + re^{i\theta}|^p + |1 - re^{i\theta}|^p
    $$
    tiene un máximo en el intervalo $[0, \frac{\pi}{2}] $ en el punto $ \theta = 0 $. Desarrollando la fórmula de Euler, $ e^{i\theta} = \cos(\theta) + i\sin(\theta)$,  y los módulos complejos tenemos
    \begin{align}
        g(\theta) &= \left|\sqrt{(1 + r \cos(\theta))^2 + (r\sin(\theta))^2} \right|^p + \left|\sqrt{(1 - r \cos(\theta))^2 + (- r\sin(\theta))^2} \right|^p \\
        &= (1 + r^2 + 2r\cos(\theta))^\frac{p}{2} + (1 + r^2 - 2r\cos(\theta))^\frac{p}{2}
    \end{align}
    Tomamos ahora la derivada $g'$ de $ g $ respecto a $ \theta $ y observamos
    \begin{align}
        g'(\theta) = &\frac{p}{2}(1 + r^2 + 2r \cos(\theta))^{\frac{p}{2}-1} (-2r \sin(\theta)) + \frac{p}{2}(1 + r^2 - 2r \cos(\theta))^{\frac{p}{2}-1} (2r \sin(\theta)) \\
        = &-pr \sin(\theta) \left((1+r^2 + 2r\cos(\theta))^{\frac{p}{2}-1} - (1+r^2 - 2r\cos(\theta))^{\frac{p}{2}-1}\right).
    \end{align}
    Como $ p \geq 2 $ entonces $g'(\theta) \leq 0 $. Es decir, la derivada de $ g $ no es creciente en todo $ \theta \in \left[0, \frac{\pi}{2}\right] $ y por tanto tiene un máximo en $ \theta = 0 $.
\end{proof}

\Clarksoni
\begin{proof}
    Podemos asumir que $ f $ y $ g $ toman valores complejos y que están definidas en casi todo punto. Por tanto, para todo $ x \in X $, tal que $ f(x) $ y $ g(x) $ estén definidas, por el Lemma \ref{lema:clarkson-2} tenemos
    \begin{equation}
        \left| z + w \right|^p + \left| z - w \right|^p \leq 2^{p-1} \left( |z|^p + |w|^p \right).
    \end{equation}
    Basta con integrar a ambos lados respecto a $ X $ para la desigualdad \eqref{eq:clarkson-1}.
\end{proof}

\begin{lemma} \label{lema:clarkson-3}
    Sean $ x \in [0, 1] $, $ 1 < p \leq 2 $ y $ q = \frac{p}{p-1} $. Se verifica la desigualdad
    \begin{align} \label{eq:clarkson-3}
        (1+x)^q + (1-x)^q \leq 2(1+x^p)^\frac{1}{p-1}.
    \end{align}
\end{lemma}
\begin{proof}
    Si $ p = 2 $, entonces $ q = 2 $ y basta desarrollar los cuadrados para comprobar que la desigualdad es cierta. Podemos limitarnos por tanto al caso $ 1 < p < 2 $. Para $ x = 0 $ y para $ x = 1 $, \eqref{eq:clarkson-3} se convierte en una igualdad. Podemos, por tanto, considerar solo $ x \in (0, 1) $. Definiendo la función $ F(u) = \frac{1 - u}{1 + u} $, tenemos que cuando $ u $ va de $0$ a $1$, $ F(u) $ decrece estrictamente de $1$ a $0$. Por tanto, \eqref{eq:clarkson-3} equivale a
    \begin{align}
        \left(1 + \frac{1 - u}{1 + u}\right)^q + \left(1 - \frac{1 - u}{1 + u}\right)^q \leq 2 \left( 1 + \left( \frac{1 - u}{1 + u} \right)^p \right)^{\frac{1}{p-1}}
    \end{align}
    para $ 0 < u < 1 $. Multiplicando ambos lados por $ (1 + u)^q $, obtenemos
    \begin{align}
        2^q (1 + u^q) \leq 2 \left( (1 + u)^p + (1 - u)^p \right)^{\frac{1}{p-1}}
    \end{align}
    Elevando ahora ambos lados a $ (p - 1) $, obtenemos
    \begin{align}
        (1 + u^q)^{p-1} \leq \frac{1}{2} \left( (1 + u)^p + (1 - u)^p \right),
    \end{align}
    para $ 0 < u < 1 $. Como estos pasos son fácilmente reversibles, es suficiente probar esta última desigualdad. Expandiendo en series de potencias, tenemos
    \begin{align}
        &\frac{1}{2}\left((1+u)^{p}+(1-u)^{p}\right)-(1+u^{q})^{p-1} \\
        &= \frac{1}{2}\left(\sum_{k=0}^{\infty}\binom{p}{k}u^{k}+\sum_{k=0}^{\infty}\binom{p}{k}(-1)^{k}\,u^{k}\right)\sum_{k=0}^{\infty}\binom{p-1}{k}u^{q\,k} \\
        &= \sum_{k=0}^{\infty}\left(\binom{p}{2\,k}u^{2\,k}-\binom{p-1}{k}u^{q\,k}\right) \\
        &= \sum_{k=1}^{\infty}\left(\binom{p}{2\,k}u^{2\,k} - \binom{p-1}{2\,k-1}u^{q\,(2\,k-1)}-\binom{p-1}{2\,k}u^{q\,2\,k}\right).
    \end{align}
    Se puede probar que esta última serie converge absoluta y uniformemente para $ u \in [0, 1] $. La prueba de ello, sin embargo, conlleva lo que en \cite{hewitt} denomina análisis {\it duro}. Las cuentas no conllevan demasiada complejidad, pero debido a su extensión, vamos a omitirlas en este trabajo. La prueba detallada se puede consultar en \cite{hewitt}[Teorema 7.25].
    
    Demostraremos que cada término $[\cdots]$ en esta serie es no negativo. Claramente, esto probará (3). El $ k $-ésimo término es
    \begin{align}
        &\frac{p\,(p-1)\,(p-2)\cdots(p-(2\,k-1))}{(2\,k)!}\,u^{2\,k} \\
        &\qquad - \frac{(p-1)\,(p-2)\cdots(p-(2\,k-1))}{(2\,k-1)!}\,u^{q\,(2\,k-1)} \\
        &\qquad - \frac{(p-1)\,(p-2)\cdots(p-2\,k)}{(2\,k)!}\,u^{q\,2\,k} \\
        &= \frac{p\,(p-1)\,(2-p)\cdots(2\,k-1-p)}{(2\,k)!}\,u^{2\,k} \\
        &\qquad - \frac{(p-1)\,(2-p)\,(3-p)\cdots(2\,k-1-p)}{(2\,k-1)!}\,u^{q\,(2\,k-1)} \\
        &\qquad + \frac{(p-1)\,(2-p)\cdots(2\,k-p)}{(2\,k)!}\,u^{q\,2\,k} \\
        &= u^{2\,k} \frac{(2-p)\,(3-p)\cdots(2\,k-p)}{(2\,k-1)!} \\
        &\qquad \times \left(\frac{p\,(p-1)}{(2\,k)\,(2\,k-p)} - \frac{(p-1)}{(2\,k-p)} u^{q\,(2\,k-1) - 2\,k} + \frac{(p-1)}{(2\,k)} u^{q\,2\,k - 2\,k}\right).
    \end{align}
    El primer factor aquí es obviamente positivo. Reescribimos la expresión entre corchetes como
    \begin{align}
        \left[\frac{1}{\frac{2\,k-p}{p-1}} - \frac{1}{\frac{2\,k}{p-1}} - \frac{1}{\frac{2\,k-p}{p-1}} u^{\frac{2\,k-p}{p-1}} + \frac{1}{\frac{2\,k}{p-1}} u^{\frac{2\,k}{p-1}}\right]
        = \left[\frac{1 - u^{\frac{2\,k-p}{p-1}}}{\frac{2\,k-p}{p-1}} - \frac{1 - u^{\frac{2\,k}{p-1}}}{\frac{2\,k}{p-1}}\right].
    \end{align}
    Un argumento elemental [que el lector deberá realizar] muestra que para cualquier $ u > 0 $ la función con valores $ \frac{1-u^{t}}{t} $, $ 0 < t < \infty $, es decreciente como función de $ t $. Como $ \frac{2\,k-p}{p-1} < \frac{2\,k}{p-1} $, se sigue que (5) es positivo.
\end{proof}

\begin{lemma} \label{lema:clarkson-4}
    Dados dos números complejos $ z, w \in \C $, si $ 1 < p \leq 2 $ tenemos
    \begin{equation} \label{eq:clarkson-4}
        |z + w|^q + |z - w|^q \leq 2(|z|^p + |w|^p)^{\frac{1}{1-p}}.
    \end{equation}
\end{lemma}
\begin{proof}
    Cuando o bien $ z = 0 $, o bien $ w=0 $ basta sustituir para verificar la desigualdad y al igual que en el Lema \ref{lema:clarkson-2} podemos suponer sin pérdida de generalidad $ 0 < |z| \leq |w| $. Así, dividiendo ambos lados entre $ |w|^q $ obtenemos
    \begin{equation}
        \left|\frac{z}{w} + 1\right|^q + \left|\frac{z}{w} - 1\right|^q \leq 2\left( \left|\frac{z}{w} \right|^p + 1\right)^{\frac{1}{1-p}}.
    \end{equation}
    Tomando exponenciales con $ \frac{z}{w} = r e^{i \theta} $ para $ 0 <r \leq 1 $ y $ 0 \leq \theta \leq 2\pi $ tenemos
    \begin{equation} \label{eq:clarkson-4-aux}
        \left|r e^{i \theta} + 1\right|^q + \left|r e^{i \theta} - 1\right|^q \leq 2\left( r^p + 1\right)^{\frac{1}{1-p}}. 
    \end{equation}
    De forma análoga al procedimiento realizado en la demostración del Lema \ref{lema:clarkson-2}, basta notar que para $ \theta = 0 $ esta última desigualdad es la probada en el Lema \ref{lema:clarkson-3}. Y de igual forma, vemos que el lado izquierdo de la desigualdad \eqref{eq:clarkson-4-aux} tiene un máximo en el intervalo $ \left[0, \frac{\pi}{2} \right] $ en el punto $ \theta = 0 $. Esto prueba entonces \eqref{eq:clarkson-4-aux} se verifica para todo $ \theta \in \left[0, 2\pi \right] $ probando así la desigualdad \eqref{eq:clarkson-4}.
\end{proof}

\Clarksonii
\begin{proof}
    Por la desigualdad Minkowski para $ 0 < p < 1 $, tenemos 
    \begin{equation} \label{eq:clarkson-minkowski}
        \left\| \left| f + g \right|^q \right\|_{p-1} + \left\| \left| f - g \right|^q \right\|_{p-1} \leq \left\| \left| f + g \right|^q + \left| f - g \right|^q \right\|_{p-1}.
    \end{equation}
    La parte izquierda de \eqref{eq:clarkson-minkowski} es igual que la parte izquierda de \eqref{eq:clarkson-2} ya que para todo $ h \in L^p(X, \mu) $ se tiene
    \begin{equation}
        \| | h |^q \|_{p-1} = \left( \int_{X} \left| |h|^q \right|^{p-1} d \mu \right)^{\frac{1}{p-1}} = \|h\|_p^q,
    \end{equation}
    ya que $ q (p-1) = p $ y $ \frac{1}{p-1} = \frac{q}{p} $. Desarrollando la norma, la parte derecha verifica
    \begin{equation}
        \left\| \left| f + g \right|^q + \left| f - g \right|^q \right\|_{p-1} = \left( \int_X \left( \left| f + g \right|^q + \left| f - g \right|^q \right)^{p-1} d \mu \right)^\frac{1}{{p-1}},
    \end{equation}
    la cual, usando el Lema \ref{lema:clarkson-4}, es menor o igual que
    \begin{equation}
        \left( \int_X 2^{p-1}( \left| f \right|^p + \left| g \right|^p)  d \mu \right)^\frac{1}{{p-1}} = 2 \left(\|f \|_p^p + \|g\|_p^p \right)^{\frac{1}{p-1}}.
    \end{equation}
\end{proof}

\begin{corollary}
    Para $ p > 1 $, el espacio $ L_p $ es uniformemente convexo.
\end{corollary}
\begin{proof}
    Sean $ f, g \in L^p $ con $ \|f\| = \|g\| = 1 $. Para $ p \geq 2 $, haciendo uso de \eqref{eq:clarkson-1} tenemos
    \begin{equation}
        \| f + g \|_p^p + \| f - g \|_p^p \leq 2^{p-1}(1 + 1) = 2^p.
    \end{equation}
    Ahora, si se tiene $\|f-g\| < \varepsilon \in (0, 2] $, dividiendo entre $ 2^p $ a ambos lados y despejando adecuadamente se sigue
    \begin{equation}
        \left\| \frac{f + g}{2} \right\|_p \leq \left( 1 - \left\| \frac{f - g}{2} \right\|_p^p\right)^{\frac{1}{p}} \leq \left( 1 - \left( \frac{\varepsilon}{2} \right)^p\right)^{\frac{1}{p}} \eqcolon \delta,
    \end{equation}
    donde $ \delta \in (0, 1) $ y se verifica la convexidad uniforme para $ p \geq 2 $. Para $ 1 < p \leq 2 $ usamos la desigualdad \eqref{eq:clarkson-2} para obtener
    \begin{equation}
        \left\| f + g \right\|_p^q + \left\| f - g \right\|_p^q \leq 2 \left(\|f \|_p^p + \|g\|_p^p \right)^{\frac{q}{p}}.
    \end{equation}
    Análogamente al procedimiento anterior, ahora con $ \delta \coloneq  \left( 1 - \left( \frac{\varepsilon}{2} \right)^q\right)^{\frac{1}{q}} $ vemos que se verifica la convexidad uniforme también para $ 1 < p \leq 2 $.
\end{proof}
