\documentclass[11pt,a4paper]{article}

\usepackage[colorlinks=true,pdftex]{hyperref}
\usepackage{latexsym}
\usepackage{amssymb}
\usepackage{amsmath}
\usepackage{amsfonts}
\usepackage{graphicx}
\usepackage{epsfig}
\usepackage{epstopdf}
\usepackage{esint}
\usepackage{hyperref}
\usepackage{amsthm}
\usepackage{mathtools}
\usepackage{color}
\usepackage[english]{babel}
\usepackage[utf8]{inputenc}
\usepackage{pdfpages}
\usepackage{enumitem}

\newcommand{\id}{\operatorname{id}}
\newcommand{\im}{\operatorname{Im}}
\newcommand{\R}{\mathbb R}
\newcommand{\Z}{\mathbb Z}
\newcommand{\C}{\mathbb C}

\addtolength{\topmargin}{-3.5cm} \addtolength{\oddsidemargin}{-2cm}
\addtolength{\textheight}{+5cm} \addtolength{\textwidth}{+4cm}

\begin{document}
\hrule\hrule
\vspace{1mm}

\noindent {\bf Curso Avanzado de Geometría, 2024/25.
\hfill{Problem Set 3: Covering Spaces}}

\vspace{1mm}

 \noindent {\bf Name}: Gonzalo Ortega Carpintero
\vspace{2mm}

\hrule\hrule

\subsection*{Exercise 1}
Let $p\colon E \to B $ be a covering map. Let $ X $ be a path connected, simply connected and locally path connected topological space, let $ x \in X $ and $ f\colon X \to B $ be a continuos surjective function.

  \begin{enumerate}[label=\alph*)]
    \item \begin{proof} By the lifting criterion we have that $ \forall e \in p^{-1}(f(x)) $, $ \exists ! \tilde f_e \colon X \to E $ continuos such that $ p \circ \tilde f_e = f $ and $ \tilde f_e(x) = e $. Suppose that $ f $ is a homeomorphism onto its image. Hence $ f $ is bijective, continuos and $ f^{-1} $ is continuos.
    
    Take $ x_1 \neq x_2 \in X $. As in particular $ f $ is injective, $ f(x_1) \neq f(x_2) $ and therefore $ p^{-1}(f(x_1)) \neq p^{-1}(f(x_1)) $. As $ p^{-1} \circ f = \tilde f_e $, we then have $ \tilde f_e(x_1) \neq \tilde f_e(x_1) $. Thus $ \tilde f_e $ is injective. Every function is surjective onto its image, so in particular, $ \tilde f_e $ is surjective onto its image and therefore bijective.
  
    By the lifting criterion, $ \tilde f_e $ is continuos. Also, $ \tilde f_e^{-1} = f^{-1} \circ p $. As $ p $ is a covering, it is continuos, and we have that, $ \tilde f_e^{-1} $ can be formed by the composition of continuos functions. Therefore, it is continuos too and $ \tilde f_e $ is a homeomorphism into its image.
    \end{proof}
  
    \item \begin{proof} Let $ e \neq e' \in p^{-1}(f(x)) $. Suppose that there exists $ h \in E $ such that $ h \in \im \tilde f_e $ and $ h \in \im \tilde f_e' $. That is, there would exist $ x, x' \in X $ such that $ \tilde f_e(x) = h $ and $ \tilde f_e'(x') = h $. Hence we have that
    $$
      f(x) = p(\tilde f_e(x)) = p(\tilde f_e'(x)) = f(x').
    $$
    As $ f $ is injective, it must be $ x = x' $. But by the uniqueness of lifts, as $ \tilde f_e $ and  $ \tilde f_e' $ coincide on one point, they must be the same, and therefore $ e = e' $ making a contradiction. Thus, the images of $ \tilde f_e $ and $ \tilde f_e' $ must be disjoint.

    Let $ h \in p^{-1} (f(X)) $. That means that $ \exists x \in X $ such that $ h = p^{-1}(f(x)) = \tilde f_h(x) \in \im(\tilde f_h)$. Thus,
    $$
      p^{-1} (f(X)) \subseteq \bigsqcup_{e \in p^{-1} (f(x))} \im(\tilde f_e).
    $$
    If $ h \in \bigsqcup_{e \in p^{-1} (f(x))} \im(\tilde f_e) $, then $ \exists e \in p^{-1} (f(x)) $ such that $ h \in \im(\tilde f_e) $. That means that $ \exists x \in X $ such that $ h = \tilde f_e(x) = p^{-1}(f(x)) \in p^{-1}(f(X)) $. Thus,
    $$
      p^{-1} (f(X)) \supseteq \bigsqcup_{e \in p^{-1} (f(x))} \im(\tilde f_e).
    $$
    \end{proof}
  \end{enumerate}


\subsection*{Exercise 5}
  Let $ /Z_8 $ act on $ S^3 = {(z,w) \in \C^2 \mid |z|^2 + |w|^2 = 1}$ as $ [m] \cdot (z, w) \coloneq (\xi^m z, \xi^m w) $, where $ \xi $ is a primitive 8-th root of unity, $ (z, w) \in S^3 $ and $ [m] \in \Z $.
  
  \begin{enumerate}[label=\alph*)]
    \item As $ \C^2 $ is a metric space, let $ d $ be the induced distance function over $ S^3 $. Define
    $$
      \epsilon \coloneq \frac{1}{2} d((z, w), \xi(z, w)).
    $$
    Let $ [m] \neq [n] \in \Z_8 $ act over $ (z, w) \in S^3 $. Let $ B_\epsilon((z, w)) $ be the open ball of center $ (z, w) $  and radius $ \epsilon $. Then $ [m] \cdot  B_\epsilon((z, w)) = B_\epsilon(\xi^m(z, w))$ and $ [n] \cdot  B_\epsilon((z, w)) = B_\epsilon(\xi^n(z, w))$. If
    $$
      B_\epsilon(\xi^m(z, w)) \cap B_\epsilon(\xi^n(z, w)) \neq \emptyset,
    $$
    then exists $ (x, y) \in S^3 $ such that $ (x, y) \in B_\epsilon(\xi^m(z, w)) $ and $ (x, y) \in B_\epsilon(\xi^n(z, w)) $. That is
    \begin{align*}
      d((x, y), \xi^m(z, w)) & < \epsilon, \\
      d((x, y), \xi^n(z, w)) & < \epsilon. \\
    \end{align*}
    But then, by the triangle inequality of metric spaces,
    $$
      d(\xi^m(z, w), \xi^n(z, w)) \leq d(\xi^m(z, w), (x, y)) + d((x, y), \xi^n(z, w)) < 2 \epsilon = d((z, w), \xi(z, w)).
    $$
    This contradicts the fact that the minimum distance between the action of two roots of unity is the distance between the action of two consecutive roots. Hence, if $ m \neq n $, then 
    $$
      B_\epsilon(\xi^m(z, w)) \cap B_\epsilon(\xi^n(z, w)) = \emptyset,
    $$
    and the action if $ \Z_8 $ over $ S^3 $ is properly discontinuos.

    \item Let $ L = S^3 / \Z_8 $. As $ \pi_1(S^3) $ is trivial, $ S^3 $ is simply connected. As $ Z_8 $ acts properly discontinuos on $ S^3$, by Corollary 16, $ \pi_1(L) \cong \Z_8 $.
    
    \item By the classification theorem for coverings, there exists a bijection between the set of coverings $p\colon E \to L $ and the conjugacy classes of groups of $ \pi_1(L)$. The sub groups of $ \Z_8 $ are: $1, \Z_2, \Z_4 $ and $ \Z_8$. As all of them are normal, the conjugacy classes are the subgroups themselves. That is, up to equivalence of coverings, the four coverings of $ S^3 $ are:
    \begin{align*}
      p_1 &\colon S^3 / \Z_8 \hookrightarrow_{\id} S^3 / \Z_8, \\
      p_2 &\colon S^3 / \Z_4 \hookrightarrow_{/ \Z_2} S^3 / \Z_8, \\
      p_3 &\colon S^3 / \Z_2 \hookrightarrow_{/ \Z_4} S^3 / \Z_8, \\
      p_4 &\colon S^3 \hookrightarrow_{/ \Z_8} S^3 / \Z_8. \\
    \end{align*}

    \item
  \end{enumerate}

\subsection*{Exercise 7}
Let $ K $ be a tame knot. Let $X$ de the 2-dimensional CW complex associated to the Wirtinger presentation of $\pi_1(X) \cong \pi_1(S^3 \setminus K)$. Let $v$ be the $0$-cell of $X$.
\begin{enumerate}[label=\alph*)]
  \item \begin{proof}
    Let $ q $ be a positive prime number. If $ H $ is a normal subgroup of $ \pi_1(X) $ of index $ q $, then $ |\pi_1(X)/H| = q $. The quotient is a finite group of prime order, hence it is a cyclic group. As the quotient is finite and cyclic, then it must be $ \pi_1(X)/H \cong \Z / q\Z $. Hence $\pi_1(X)/H$ is unique up to isomorphism and $ H $ would be the unique subgroup generating the quotient map $ \phi \colon \pi_1(X) \twoheadrightarrow \pi_1(X) / H \cong \Z / q\Z $.

    To prove the existence of $H$, note that as $ K $ is tame, Wirtinger presentation of $\pi_1(X)$ has a finite number of generators $ g_1, \dots, g_n$. Fix any $[k] \in \Z / q\Z $ different from $ 0 $ and note that it must be a generator of $ \Z / q\Z $ since $ q $ is prime. Map every generator of $\pi_1(X)$ to $ [k] $ trough the homomorphism $ \phi \colon \pi_1(X) \twoheadrightarrow \Z / q\Z $. That is, $ \phi(g_i) = [k] $ for all $ i = 1 \dots n $. This is well-defined because the Wirtinger presentation relations imply that all generators are conjugate to one another, and they are sent to the same element under the given assignment. Note that $ \phi $ is surjective because $ h $ is a generator. By the first group isometry theorem we have
    $$
      \pi_1(X)/\ker(\phi) \cong \im(\phi) \cong \Z / q\Z.
    $$
    Now, just take $ H = \ker(\phi) $.
  \end{proof}
  \item Let $ K = 4_1 $. Let $ p: Y \to X $ be the covering map associate to $\pi_1(X, v) $. Let $ q = 2 $. As seen in class, the Wirtinger of $ \pi_1(S^3/4_1) $ is
  $$
    \pi_1(S^3/4_1) \cong \langle a, b, c, d \mid a c^{-1} b^{-1} c, b d^{-1} c^{-1} d, c a^{-1} d^{-1} a \rangle.
  $$
  Following the procedure seen in class, $ Y^0 $ are the elements of $ \Z / 2\Z $:
  $$
    Y^0 = \{ [0], [1] \}.
  $$
  The elements of $Y^1$ are in bijection with $ \{ [0], [1] \} \times \{a, b, c, d\} $:
  \begin{align*}
    Y^1 = \bigl\{&e^1_{(0, a)} \eqcolon e, e^1_{(0, b)} \eqcolon f, e^1_{(0, c)} \eqcolon g, e^1_{(0, d)} \eqcolon h, \\
    &e^1_{(1, a)} \eqcolon i, e^1_{(1, b)} \eqcolon j, e^1_{(1, c)} \eqcolon k, e^1_{(1, d)} \eqcolon l\bigr\}.
  \end{align*}
  Finally, elements in $Y^1$ are in bijection with $ \{ [0], [1] \} \times \{a c^{-1} b^{-1} c, b d^{-1} c^{-1} d, c a^{-1} d^{-1} a\} $:
  \begin{align*}
    Y^2 = \bigl\{&e^2_{(0, a c^{-1} b^{-1} c)}, e^2_{(0, b d^{-1} c^{-1} d)}, e^2_{(0, c a^{-1} d^{-1} a)}, \\
    &e^2_{(1, a c^{-1} b^{-1} c)}, e^2_{(1, b d^{-1} c^{-1} d)}, e^2_{(1, c a^{-1} d^{-1} a)}\bigr\}.
  \end{align*}
  where the gluing maps can be given by
  \begin{align*}
    e^2_{(0, a c^{-1} b^{-1} c)} &\longrightarrow e g^{-1} j^{-1} k \\
    e^2_{(1, a c^{-1} b^{-1} c)} &\longrightarrow i k f g \\
    \cdots
  \end{align*}
  \end{enumerate}
  

\begin{thebibliography}{9}

  \bibitem{gath}
  Allen Hatcher,
  \textit{Algebraic Topology},
  Allen Hatcher 2001.

\end{thebibliography}

\end{document}
