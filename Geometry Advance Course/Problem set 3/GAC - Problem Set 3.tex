\documentclass[11pt,a4paper]{article}

\usepackage[colorlinks=true,pdftex]{hyperref}
\usepackage{latexsym}
\usepackage{amssymb}
\usepackage{amsmath}
\usepackage{amsfonts}
\usepackage{graphicx}
\usepackage{epsfig}
\usepackage{epstopdf}
\usepackage{esint}
\usepackage{hyperref}
\usepackage{amsthm}
\usepackage{mathtools}
\usepackage{color}
\usepackage[english]{babel}
\usepackage[utf8]{inputenc}
\usepackage{pdfpages}
\usepackage{enumitem}

\newcommand{\id}{\operatorname{id}}
\newcommand{\im}{\operatorname{Im}}
\newcommand{\R}{\mathbb R}
\newcommand{\Z}{\mathbb Z}

\addtolength{\topmargin}{-3.5cm} \addtolength{\oddsidemargin}{-2cm}
\addtolength{\textheight}{+5cm} \addtolength{\textwidth}{+4cm}

\begin{document}
\hrule\hrule
\vspace{1mm}

\noindent {\bf Curso Avanzado de Geometría, 2024/25.
\hfill{Problem Set 3: Covering Spaces}}

\vspace{1mm}

 \noindent {\bf Name}: Gonzalo Ortega Carpintero
\vspace{2mm}

\hrule\hrule

\subsection*{Exercise 1}
Let $p\colon E \to B $ be a covering map. Let $ X $ be a path connected, simply connected and locally path connected topological space, let $ x \in X $ and $ f\colon X \to B $ be a continuos surjective function.

  \begin{enumerate}[label=\alph*)]
    \item \begin{proof} By the lifting criterion we have that $ \forall e \in p^{-1}(f(x)) $, $ \exists ! \tilde f_e \colon X \to E $ continuos such that $ p \circ \tilde f_e = f $ and $ \tilde f_e(x) = e $. Suppose that $ f $ is a homeomorphism onto its image. Hence $ f $ is bijective, continuos and $ f^{-1} $ is continuos.
    
    Take $ x_1 \neq x_2 \in X $. As in particular $ f $ is injective, $ f(x_1) \neq f(x_2) $ and therefore $ p^{-1}(f(x_1)) \neq p^{-1}(f(x_1)) $. As $ p^{-1} \circ f = \tilde f_e $, we then have $ \tilde f_e(x_1) \neq \tilde f_e(x_1) $. Thus $ \tilde f_e $ is injective. Every function is surjective onto its image, so in particular, $ \tilde f_e $ is surjective onto its image and therefore bijective.
  
    By the lifting criterion, $ \tilde f_e $ is continuos. Also, $ \tilde f_e^{-1} = f^{-1} \circ p $. As $ p $ is a covering, it is continuos, and we have that, $ \tilde f_e^{-1} $ can be formed by the composition of continuos functions. Therefore, it is continuos too and $ \tilde f_e $ is a homeomorphism into its image.
    \end{proof}
  
    \item \begin{proof} Let $ e \neq e' \in p^{-1}(f(x)) $. Suppose that there exists $ h \in E $ such that $ h \in \im \tilde f_e $ and $ h \in \im \tilde f_e' $. That is, there would exist $ x, x' \in X $ such that $ \tilde f_e(x) = h $ and $ \tilde f_e'(x') = h $. Hence we have that
    $$
      f(x) = p(\tilde f_e(x)) = p(\tilde f_e'(x)) = f(x').
    $$
    As $ f $ is injective, it must be $ x = x' $. But by the uniqueness of lifts, as $ \tilde f_e $ and  $ \tilde f_e' $ coincide on one point, they must be the same, and therefore $ e = e' $ making a contradiction. Thus, the images of $ \tilde f_e $ and $ \tilde f_e' $ must be disjoint.

    Let $ h \in p^{-1} (f(X)) $. That means that $ \exists x \in X $ such that $ h = p^{-1}(f(x)) = \tilde f_h(x) \in \im(\tilde f_h)$. Thus,
    $$
      p^{-1} (f(X)) \subseteq \bigsqcup_{e \in p^{-1} (f(x))} \im(\tilde f_e).
    $$
    If $ h \in \bigsqcup_{e \in p^{-1} (f(x))} \im(\tilde f_e) $, then $ \exists e \in p^{-1} (f(x)) $ such that $ h \in \im(\tilde f_e) $. That means that $ \exists x \in X $ such that $ h = \tilde f_e(x) = p^{-1}(f(x)) \in p^{-1}(f(X)) $. Thus,
    $$
      p^{-1} (f(X)) \supseteq \bigsqcup_{e \in p^{-1} (f(x))} \im(\tilde f_e).
    $$
    \end{proof}
  \end{enumerate}


\subsection*{Exercise 5}

\subsection*{Exercise 7}

\begin{thebibliography}{9}

  \bibitem{gath}
  Allen Hatcher,
  \textit{Algebraic Topology},
  Allen Hatcher 2001.

\end{thebibliography}

\end{document}
