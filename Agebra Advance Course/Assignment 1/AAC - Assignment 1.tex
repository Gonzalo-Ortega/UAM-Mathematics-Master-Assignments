\documentclass[11pt,a4paper]{article}

\usepackage[colorlinks=true,pdftex]{hyperref}
\usepackage{latexsym}
\usepackage{amssymb}
\usepackage{amsmath}
\usepackage{amsfonts}
\usepackage{graphicx}
\usepackage{epsfig}
\usepackage{epstopdf}
\usepackage{esint}
\usepackage{hyperref}
\usepackage{amsthm}
\usepackage{mathtools}
\usepackage{color}
\usepackage[english]{babel}
\usepackage[utf8]{inputenc}

\newcommand{\id}{\operatorname{id}}
\newcommand{\R}{\mathbb R}
\newcommand{\Z}{\mathbb Z}
\newcommand{\Q}{\mathbb Q}
\newcommand{\N}{\mathbb N}

\addtolength{\topmargin}{-3.5cm} \addtolength{\oddsidemargin}{-2cm}
\addtolength{\textheight}{+5cm} \addtolength{\textwidth}{+4cm}

\begin{document}
\hrule\hrule
\vspace{1mm}

\noindent {\bf Algebra Advance Course, 2024/25.
\hfill{Assignment 1}}

\vspace{1mm}

 \noindent {\bf Name}: Gonzalo Ortega Carpintero
\vspace{2mm}

\hrule\hrule

\subsection*{Exercise 1}
\begin{enumerate}
  \item For each $ n \geq 0 $, define the set $ S_n = \{p q p^{-1}, p^2 q p^{-2}, \dots, p^2 q p^{-n}\} $ and the function $ \theta \colon S_n \to F_2 = \langle a, b \mid -\rangle $ that carries each $ p^i q p^{-i} $ to $ \theta(p^i q p^{-i}) = a^i b a^{-i} $. The set $ S_n $ has $ n $ elements so the free group generated by $ S_n $, $F(S_n)$, is isomorphic to $ F_n $. By the universal property of free groups, there is a unique homomorphism $ \hat \theta \colon F(S_n) \to F_2 $ extending $ \theta $. Clearly, the restriction to its image $ \hat \theta \colon F(S_n) \to \hat \theta (F(S_n)) $ is a bijection, so $ F_n \cong \hat \theta(F(S_n)) $. We just need to prove that $ \hat \theta(F(S_n)) $ is a subgroup of $ F_2 $.

  Let $ x = \prod_{i=0}^{l}(p^{\alpha_i} q^{\epsilon_i} p^{-\alpha_i}) $,
  $ y = \prod_{j=0}^{m}(p^{\beta_j} q^{\delta_j} p^{-\beta_j}) $, with $l, m \in \N $, $\alpha_i, \beta_j \in \Z $, $\epsilon_i, \delta_j \in \{ \pm 1\} $ be two arbitrary elements of $ \hat \theta( F(S_n))$. Then
  $$
    xy^{-1} = \prod_{i=0}^{l}(a^{\alpha_i} b^{\epsilon_i} a^{-\alpha_i})
    \prod_{j=0}^{m}(a^{\beta_j} b^{-\delta_j} a^{-\beta_j}) = \prod_{k=0}^{l + m} (a^{\gamma_k} b^{\xi_k} a^{-\gamma_k})  
  $$
  with $ \gamma_k \in \Z $, $ \xi_k \in \{ \pm 1\} $. Thus $ xy^{-1} \in \hat \theta(F(S_n)) $ and $ \hat \theta(F(S_n)) $ is a subgroup of $ F_2 $.

  \item To prove that $ F_2 $ contains a subgroup isomorphic to $ F_\infty $ its enough to take the set $ S_\infty = \{p^i q p^{-i} : i \in \N \} $ and repeat the steps followed in 1.
\end{enumerate}


\subsection*{Exercise 2}
Let $ \Q^* = \Q - \{0\} $ be the multiplicative group of non-zero rationals. Every element $ q \in \Q^* $ can be written as $ q = \frac{a}{b} $ with $ a, b \in \Z $. Every integer admits a prime factorization such that $ a = p_1^{e_1} \dots p_n^{e_n} $, $ b = q_1^{f_1} \dots q_m^{f_m} $ with $p_i, q_i $ prime numbers and $ e_i, f_k \in \N$. If $Q*$ were finitely generated, there would be a finite set $ S $ which generates $Q*$. Each element  $ \frac{a}{b} \in S $ could be decompose into a fraction of prime factorizations. But prime numbers are infinite, so piking a prime $ p $ not included in any of the factorizations of the elements in $ S $ would be a contradiction. As $ p \in \Q^*$ but it can not be generated by elements of $ S $ as it is a prime not in $ S $. Thus, $ \Q^*$ must be infinitely generated.

\subsection*{Exercise 3}
Let $ G = \langle S \mid R \rangle $ and $ H = \langle T \mid Q \rangle $. Define $ \theta: S \sqcup T \to G \times H $ such that $ \theta(s) = (s, 1) $ and $ \theta(s) = (1, t) $ for all $ s \in S, t\in T$. By the universal property of free groups, there is a unique $ \hat \theta \colon F(S \sqcup T) \to G \times H $. As $ R , Q , [S, T] \subset \ker(\hat \theta) $, also $ \langle \langle R \sqcup Q \sqcup [S, T]\rangle\rangle \subset \ker(\hat \theta) $. Thus, by the fundamental theorem on homomorphisms, there exists a unique surjective homomorphism $ f \colon F(S \sqcup T) / \langle \langle R \sqcup Q \sqcup [S, T]\rangle\rangle \to G \times H $. Its easy to check that $ f $ is also injective, making
$$
  G \times H = \langle S \sqcup T \mid R \sqcup Q \sqcup [S, T] \rangle.
$$

Now if $ G $ and $ H $ are finitely presented, then $ |S|, |R|, |T|, |Q| < \infty $. Therefore $ |S \sqcup T|, |R \sqcup Q|, [S, T] < \infty $ and $ G \times H $ is finitely presented.

If $ G \times H $ is finitely presented then it has a presentation $ \langle X | Y \rangle $ such that $|X|, |Y| < \infty $. Taking the projections $ \phi \colon G \times H \to G $ and $ \psi \colon G \times H \to H $. Therefore, $ G = \langle \phi(X) | R \rangle $ and $ G = \langle \psi(X) | Q \rangle $ were $ R $ and $ Q $ are two unknown pair of restriction sets. By the first part of this exercise, $ G \times H $ can also be presented as
$$
  G \times H = \langle \phi(X) \sqcup \psi(X) | R \sqcup Q \sqcup [\phi(X), \psi(X)] \rangle.
$$
We have seen in class that $ Y $ is finite there must exist some $Y' \subseteq R \sqcup Q \sqcup [\phi(X), \psi(X)] $ which is also finite and 
$$
  G \times H = \langle \phi(X) \sqcup \psi(X) | Y' \rangle.
$$
But it needs to be $ Y' \cap R = R $ and $ Y' \cap Q = Q $ so $ G $ and $ H $ are also finitely presented.

\subsection*{Exercise 4}
Let $ G_1 = \langle S_1 = \{a, b\} \mid R_1 = \{a^3 b^5 a^{-3} b^{-5}\} \rangle $. To show that $ G_1 $ is infinite we will construct a surjective homomorphism from $ G_1 $ to $ \Z $. First, define $ \theta \colon S_1 \to \Z $ as $ \theta(a) = 2 $ and $ \theta(b) = -1 $. By the universal property of free groups there is a unique homomorphism $ \hat \theta \colon F(S) \to \Z $ extending $ \theta $ such that $ \hat \theta(a) = 2 $ and $ \hat \theta(b) = -1 $. Thus, we have that
$$
  \hat \theta (a^3 b^5 a^{-3} b^{-5}) = 3 \hat\theta(a) + 5 \hat\theta(b) - 3 \hat\theta(a) - 5 \hat\theta(b) = 0.
$$
Thus, $ R_1 \subset \ker (\hat \theta) $ and in fact $ \langle \langle R_1 \rangle \rangle \subset \ker (\hat \theta) $. For $ (ab)^n \in F(S_1)$, $ n \in \Z $, we have that $ \hat \theta((ab)^n) = n $. This makes any element of $\Z$ reachable from an element of $ F(S_1) $ by $ \hat \theta $, making $ \hat \theta $ a surjective homomorphism. Hence, by the fundamental theorem on homomorphisms, there exists a unique surjective homomorphism $ h_1: F(S_1)/\langle \langle R_1 \rangle \rangle = G_1 \to \Z $ proving $ G_1 $ is infinite.

Let now $ G_2 = \langle S_2 = \{a, b\} \mid R_2 = \{a^2 b^3\} \rangle $. We proceed as before. Define $ \phi \colon S_2 \to \Z $ as $ \phi(a) = 3 $ and $ \phi(b) = -2 $. Then there exists a unique homomorphism $ \hat \phi \colon F(S_2) \to \Z $ such that $ \hat \phi(a) = 3 $ and $ \hat \phi(b) = -2 $. Then
$$
  \hat \phi (a^2 b^3) = 3 \hat\phi(a) -2 \hat\phi(b) = 3 \cdot (-2) -2 \cdot 3 = 0.
$$
Hence, $ \langle \langle R_2 \rangle \rangle \subset \ker (\hat \phi) $ and $ \hat \phi((ab)^n) = n $,  $ \phi $ is surjective and there exists a unique surjective $ h_2: F(S_2)/\langle \langle R_2 \rangle \rangle = G_2 \to \Z $. Thus, $ G_2 $ is also infinite.

\subsection*{Exercise 8}
Let $ G = \langle S \mid R \rangle = F(S) / \langle \langle R \rangle \rangle $ be a finite presentation. All words $ w \in (S \sqcup S^{-1})^* $ such that $ w = 1 $ in $ G $ are the words $ w \in \langle \langle R \rangle \rangle $ by the definition of group presentation. Recall that
$$
  \langle \langle R \rangle \rangle = \bigcup_{i=0}^\infty \left\{ \prod_{j=0}^\infty (g_j^{-1} r_j^{\epsilon_j} g_j) \mid g_j \in F(G), r_j \in R, \epsilon_j \in \{ \pm 1 \} \right\}.
$$
To enumerate the words $ w $ we can proceed as follows:
\begin{enumerate}
  \item As $ |R| $ is finite, suppose $ |R| = n $. We can enumerate all elements of $ R $ and $ R^{-1} $ numbering them as:
  \begin{align} \label{R}
    r_1, r_1^{-1}, r_2, r_2^{-1}, \dots, r_n, r_n^{-1}.
  \end{align}
  
  \item In the same manner, as $ |S| $ is finite, suppose $ |S| = m $, and enumerate all elements of $ S $ and $ S^{-1} $ as:
  \begin{align} \label{S}
    s_1, s_1^{-1}, s_2, s_2^{-1}, \dots, s_m, s_m^{-1}.
  \end{align}

  \item Finally, now we just need to enumerate the elements of $ \langle \langle R \rangle \rangle $ in a sorted way without enumerating one same element more than once. For so, start enumerating the elements $ g \in F(S) $ by making combinations of the elements of \eqref{S} in a lexicographic order and in increasing word length. As $|S|$ is finite, for each word length $ k $, the amount of words of $ F(S) $ of length $ k $ is going to be $ m ^k $ minus the number of produced words that can be reduced. In any case, there is a finite number of words of length $ k $ in $ F(S) $. Denote the set of words of length less than or equal to $ k $ as  $ F(S)_k $ and note that it is finite too.
  
  For each word length $ k $, we can iterate over the elements of \eqref{R}, and enumerate all the elements
  $$
  \prod_{j=0}^k (g_j^{-1} r_j^{\epsilon_j} g_j) \text{ with } g_j \in F(S)_k, r_j \in R, \epsilon_j \in \{ \pm 1 \}.
  $$

  We reduce each obtained word and compare it with the finite number of words we had previously enumerated. If it is a new word, we enumerate it. 
\end{enumerate}

Each $k$-th iteration of Step 3 of the previous procedure is finite as \eqref{R} is finite and $ F(S)_k $ is finite. Therefore, on an input $ w \in (S \sqcup S^{-1})^* $, if $ w = 1 $ in $ G $, as $ w $ would have finite length, our procedure will find it in finite time. Else, our procedure may run forever.

\begin{thebibliography}{9}

  \bibitem{marco}
  Marco Linton,
  \textit{Geometric group theory notes},
  UAM Algebra Advance Course, 2025.
  
\end{thebibliography}

\end{document}
